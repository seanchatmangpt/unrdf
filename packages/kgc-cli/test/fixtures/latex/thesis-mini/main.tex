% Thesis mini fixture - tests multi-file compilation and cross-references
\documentclass{article}

% Load preamble (tests \input resolution)
% Minimal preamble for thesis-mini fixture
% Provides custom macros and document configuration

% Custom commands for testing \input resolution
\newcommand{\testmacro}[1]{\textbf{#1}}
\newcommand{\kgc}{\textsc{KGC}}
\newcommand{\unrdf}{\textsc{unrdf}}

% Custom environments
\newenvironment{theorem}[1][Theorem]{%
  \par\medskip\noindent\textbf{#1.}~\itshape%
}{%
  \par\medskip%
}

% Section formatting (minimal)
\setcounter{secnumdepth}{3}
\setcounter{tocdepth}{2}

% Testing cross-reference setup
\makeatletter
\newcommand{\chapterref}[1]{Chapter~\ref{#1}}
\newcommand{\sectionref}[1]{Section~\ref{#1}}
\makeatother


\title{Multi-File Test Document}
\author{Test Suite}
\date{\today}

\begin{document}

\maketitle

\tableofcontents

\section{Introduction}
\label{sec:intro}

This document tests multi-file compilation with cross-references and
includes. It demonstrates the \testmacro{custom macro system} defined
in \texttt{preamble.tex}.

The \kgc{} system compiles \LaTeX{} using \unrdf{} infrastructure.

\section{Cross-Reference Testing}
\label{sec:crossref}

This section (\sectionref{sec:crossref}) references the introduction
(\sectionref{sec:intro}). Cross-references require multiple compilation
passes to resolve correctly.

\subsection{Equations}
\label{subsec:eq}

Consider the following equation:
\begin{equation}
  E = mc^2
  \label{eq:einstein}
\end{equation}

Equation~\ref{eq:einstein} demonstrates mass-energy equivalence.

\subsection{Theorems}

\begin{theorem}[Test Theorem]
  This is a test theorem using a custom environment.
\end{theorem}

\section{Multi-Pass Verification}
\label{sec:multipass}

The following items require multi-pass compilation:

\begin{itemize}
  \item Table of contents generation
  \item Cross-reference resolution (e.g., \sectionref{sec:intro})
  \item Label numbering consistency
  \item Equation references (e.g., Equation~\ref{eq:einstein})
\end{itemize}

\section{Conclusion}
\label{sec:conclusion}

This document validates that the compilation pipeline:

\begin{enumerate}
  \item Loads files from multiple sources via \textbackslash input
  \item Executes multiple compilation passes
  \item Resolves all cross-references correctly
  \item Generates a complete table of contents
\end{enumerate}

See \sectionref{sec:intro} for context and \sectionref{sec:crossref}
for methodology.

\end{document}
