% agent_8_packages.tex
% Agent 8 Package Documentation: Packages 31-36 (Application Layer)
% Generated: 2026-01-11

% ============================================================================
% Package 31: @unrdf/cli
% ============================================================================

\label{pkg:unrdf-cli}
\section{\pkg{@unrdf/cli} --- Command-Line Interface}

\begin{pkgmeta}
Path & \texttt{packages/cli} \\
Kind & js \\
Entrypoints & 2 files \\
Dependencies & 6 (citty, table, yaml, @unrdf/core, @unrdf/federation, @unrdf/streaming) \\
Blurb & Command-line tools for graph operations, SPARQL queries, and context management \\
\end{pkgmeta}

\subsection*{Observable \(\Oobs\) and Artifact \(\Aout\)}

The CLI observes user commands and filesystem state:

\[
\Oobs_{\text{cli}} = \{ \text{commands}, \text{args}, \text{stdin}, \text{files}_{\text{rdf}} \}
\]

Artifacts include command execution results and formatted output:

\[
\Aout_{\text{cli}} = \{ \text{tables}, \text{json}, \text{yaml}, \text{receipts}, \text{exports} \}
\]

Primary observables:
\begin{itemize}
\item Command-line arguments via \texttt{citty} framework
\item RDF files for parsing and validation
\item Store state for backup/restore operations
\item SPARQL query strings from stdin or files
\end{itemize}

Artifacts produced:
\begin{itemize}
\item Formatted query results (table, JSON, YAML)
\item Receipt verification reports
\item Store backups with checksums
\item Import/export manifests
\end{itemize}

\subsection*{Type Signature \(\SigmaType\)}

Command signature from \texttt{packages/cli/src/cli.mjs}:

\begin{lstlisting}[language=JavaScript]
// Command definition schema (citty framework)
const CommandSchema = {
  meta: {
    name: z.string(),
    description: z.string(),
    version: z.string()
  },
  args: z.record(z.object({
    type: z.enum(['string', 'boolean', 'positional']),
    description: z.string(),
    required: z.boolean().optional()
  }))
};
\end{lstlisting}

Core command types:
\begin{itemize}
\item \texttt{query}: Execute SPARQL queries against store
\item \texttt{import}: Load RDF data from files
\item \texttt{export}: Dump store to various formats
\item \texttt{backup}: Create versioned store snapshots
\item \texttt{restore}: Restore from backup with integrity check
\item \texttt{receipts}: Verify receipt chains
\end{itemize}

Type preservation through command pipeline:
\[
\SigmaType(\text{stdin}) \to \SigmaType(\text{parse}) \to \SigmaType(\text{execute}) \to \SigmaType(\text{format})
\]

\subsection*{Reconciler \(\muRecon\)}

The CLI reconciler maps commands to RDF operations:

\[
\muRecon_{\text{cli}}: \Oobs_{\text{cmd}} \to \Aout_{\text{result}}
\]

Command execution pipeline from \texttt{store-import.mjs}:
\begin{enumerate}
\item \texttt{parseArgs()}: Command-line $\to$ Validated options
\item \texttt{loadRDFFiles()}: File paths $\to$ RDF quads
\item \texttt{importToStore()}: Quads $\to$ Store mutations
\item \texttt{generateReceipt()}: Store state $\to$ Cryptographic proof
\end{enumerate}

Backup reconciler from \texttt{store-backup.mjs}:
\[
\muRecon_{\text{backup}}: \store \to \{ \text{snapshot}, \text{checksum}, \text{timestamp} \}
\]

Implementation:
\begin{lstlisting}[language=JavaScript]
async function createBackup(store, options) {
  const timestamp = toISO(now());
  const quads = await store.match(null, null, null, null);
  const serialized = serializeQuads(quads, options.format);
  const checksum = blake3Hash(serialized);

  return {
    version: '1.0',
    timestamp,
    format: options.format,
    checksum,
    quadCount: quads.length,
    data: serialized
  };
}
\end{lstlisting}

Restore reconciler ensures integrity:
\begin{lstlisting}[language=JavaScript]
async function verifyAndRestore(backup) {
  const computedHash = blake3Hash(backup.data);
  if (computedHash !== backup.checksum) {
    throw new Error('Backup integrity check failed');
  }

  await store.load(backup.data, backup.format);
  return { restored: backup.quadCount };
}
\end{lstlisting}

\subsection*{Composition \(\PiMerge / \oplusMerge\)}

Sequential command pipeline:
\[
\PiMerge(\text{import}, \text{query}, \text{export}) = \text{ETL Pipeline}
\]

Example composition:
\begin{lstlisting}[language=Bash]
# Import RDF data
unrdf import --file data.ttl --format turtle

# Query transformed data
unrdf query --sparql 'SELECT * WHERE { ?s ?p ?o } LIMIT 10'

# Export results
unrdf export --output results.json --format json
\end{lstlisting}

Parallel execution with streaming:
\[
\Aout_{\text{total}} = \bigoplus_{f \in \text{Files}} \muRecon_{\text{import}}(f)
\]

Streaming composition via \texttt{@unrdf/streaming}:
\begin{itemize}
\item Multi-file import with progress tracking
\item Incremental receipt generation
\item Parallel validation of backup checksums
\end{itemize}

\subsection*{Guard \(\GuardH\) and Invariant \(\InvQ\)}

Guards prevent invalid operations:
\begin{itemize}
\item \(\GuardH_{\text{file}}\): File must exist before import
\item \(\GuardH_{\text{format}}\): RDF format must be supported (turtle, ntriples, nquads, trig)
\item \(\GuardH_{\text{checksum}}\): Backup checksum must match before restore
\item \(\GuardH_{\text{query}}\): SPARQL syntax must be valid
\end{itemize}

Invariants preserved:
\begin{itemize}
\item \(\InvQ_{\text{backup}}\): Backup creation is deterministic for given store state
\item \(\InvQ_{\text{restore}}\): Restore produces equivalent store to backup source
\item \(\InvQ_{\text{idempotent}}\): Re-importing same file produces same store state
\item \(\InvQ_{\text{receipt}}\): Receipt chain remains valid across backup/restore
\end{itemize}

Backup integrity invariant:
\[
\InvQ_{\text{integrity}}(\text{backup}) \iff \text{blake3}(\text{backup.data}) = \text{backup.checksum}
\]

\subsection*{Provenance and Receipts}

CLI operations generate receipts from \texttt{cli-receipts.mjs}:
\begin{lstlisting}[language=JavaScript]
{
  operation: "import",
  timestamp: "2026-01-11T00:00:00.000Z",
  files: ["data.ttl"],
  quadsAdded: 1234,
  format: "turtle",
  checksums: {
    "data.ttl": "blake3:abc123..."
  },
  receiptHash: "blake3:def456..."
}
\end{lstlisting}

Provenance chain for backup operations:
\begin{itemize}
\item Source store state hash
\item Backup creation timestamp
\item Format and compression options
\item Quad count at backup time
\item Restore verification status
\end{itemize}

Receipt verification command:
\begin{lstlisting}[language=Bash]
unrdf receipts verify --file backup-receipt.json
\end{lstlisting}

\subsection*{Minimal Example}

\begin{lstlisting}[language=Bash]
# Import RDF ontology
unrdf import --file ontology.ttl --format turtle

# Execute SPARQL query
unrdf query --sparql 'SELECT ?s ?p ?o WHERE {
  ?s a <http://example.org/Person> .
  ?s ?p ?o .
} LIMIT 5' --format table

# Create backup with receipt
unrdf backup --output snapshot.nq --format nquads --receipt

# Verify backup integrity
unrdf restore --dry-run --file snapshot.nq
\end{lstlisting}

\subsection*{Open Questions}

\begin{enumerate}
\item Can streaming import scale to multi-GB RDF files with constant memory?
\item How to handle partial restore failures with transactional rollback?
\item What is optimal buffer size for parallel file import?
\item Can receipt chains survive across multiple backup/restore cycles?
\end{enumerate}

% ============================================================================
% Package 32: @unrdf/yawl-api
% ============================================================================

\label{pkg:unrdf-yawl-api}
\section{\pkg{@unrdf/yawl-api} --- REST API Framework}

\begin{pkgmeta}
Path & \texttt{packages/yawl-api} \\
Kind & js \\
Entrypoints & 1 file \\
Dependencies & 7 (fastify, @fastify/swagger, @fastify/swagger-ui, zod, @unrdf/yawl, @unrdf/kgc-4d) \\
Blurb & High-performance REST API framework exposing YAWL workflows with OpenAPI documentation \\
\end{pkgmeta}

\subsection*{Observable \(\Oobs\) and Artifact \(\Aout\)}

Observables are HTTP requests and workflow state:

\[
\Oobs_{\text{api}} = \{ \text{requests}, \text{routes}, \text{workflows}, \text{cases}_{\text{state}} \}
\]

Artifacts include HTTP responses with HATEOAS links:

\[
\Aout_{\text{api}} = \{ \text{responses}, \text{receipts}, \text{links}_{\text{HATEOAS}}, \text{OpenAPI}_{\text{spec}} \}
\]

The server observes:
\begin{itemize}
\item HTTP requests validated by Zod schemas
\item YAWL engine events (task enabled, started, completed)
\item Workflow definitions for schema generation
\item Case state transitions for hypermedia links
\end{itemize}

Produces:
\begin{itemize}
\item RESTful endpoints for workflow lifecycle
\item OpenAPI 3.1 specification with Swagger UI
\item HATEOAS hypermedia controls
\item Workflow execution receipts
\end{itemize}

\subsection*{Type Signature \(\SigmaType\)}

Request validation schemas from \texttt{server.mjs}:

\begin{lstlisting}[language=JavaScript]
const CreateCaseRequestSchema = z.object({
  workflowId: z.string(),
  initialData: z.record(z.any()).optional(),
  caseId: z.string().optional()
});

const CompleteTaskRequestSchema = z.object({
  actor: z.string().optional(),
  output: z.record(z.any()).optional()
});

const WorkflowDefinitionSchema = z.object({
  id: z.string(),
  name: z.string().optional(),
  version: z.string().default('1.0.0'),
  tasks: z.array(z.record(z.any())),
  flows: z.array(z.record(z.any())).optional()
});
\end{lstlisting}

Server configuration type:
\[
\SigmaType_{\text{server}}: \{ \text{engine}, \text{baseUrl}, \text{fastifyOptions} \} \to \text{FastifyInstance}
\]

HATEOAS link signature:
\[
\SigmaType_{\text{links}}: \text{CaseState} \to \{ \text{self}, \text{workflow}, \text{enabledTasks} \}
\]

\subsection*{Reconciler \(\muRecon\)}

The reconciler maps HTTP requests to workflow operations:

\[
\muRecon_{\text{api}}: \Oobs_{\text{HTTP}} \to \Aout_{\text{workflow}}
\]

Request processing pipeline:
\begin{enumerate}
\item Fastify route handler receives request
\item Zod schema validates request body
\item YAWL engine executes workflow operation
\item HATEOAS link generator creates hypermedia controls
\item Response serializer formats output
\end{enumerate}

HATEOAS link generation from \texttt{server.mjs}:
\begin{lstlisting}[language=JavaScript]
function generateHATEOASLinks(caseInstance, baseUrl) {
  const links = {
    self: {
      href: `${baseUrl}/api/cases/${caseInstance.id}`,
      method: 'GET'
    },
    workflow: {
      href: `${baseUrl}/api/workflows/${caseInstance.workflowId}`,
      method: 'GET'
    },
    enabledTasks: []
  };

  for (const [workItemId, workItem] of caseInstance.workItems) {
    if (workItem.status === 'enabled') {
      links.enabledTasks.push({
        taskId: workItem.taskId,
        workItemId,
        name: workItem.name,
        actions: {
          start: {
            href: `${baseUrl}/api/cases/${caseInstance.id}/tasks/${workItemId}/start`,
            method: 'POST'
          },
          cancel: {
            href: `${baseUrl}/api/cases/${caseInstance.id}/tasks/${workItemId}/cancel`,
            method: 'POST'
          }
        }
      });
    }
  }

  return links;
}
\end{lstlisting}

Endpoint reconcilers:
\begin{itemize}
\item \texttt{POST /api/workflows/:id/cases} $\to$ \texttt{engine.createCase()}
\item \texttt{POST /api/cases/:id/tasks/:wiId/start} $\to$ \texttt{engine.startTask()}
\item \texttt{POST /api/cases/:id/tasks/:wiId/complete} $\to$ \texttt{engine.completeTask()}
\item \texttt{GET /api/cases/:id} $\to$ Case state + HATEOAS links
\end{itemize}

\subsection*{Composition \(\PiMerge / \oplusMerge\)}

Middleware pipeline composition:
\[
\PiMerge(\text{CORS}, \text{Validation}, \text{Execution}, \text{HATEOAS}) = \text{Endpoint}
\]

Fastify plugin composition:
\begin{lstlisting}[language=JavaScript]
const server = await createYAWLAPIServer({
  engine: createWorkflowEngine(),
  baseUrl: 'http://localhost:3000',
  enableSwagger: true
});

// Plugins composed via Fastify's plugin system
server.register(fastifyCors);
server.register(fastifySwagger, swaggerOptions);
server.register(fastifySwaggerUI, uiOptions);
\end{lstlisting}

Parallel request processing:
\[
\Aout_{\text{responses}} = \bigoplus_{r \in \text{Requests}} \muRecon_{\text{handle}}(r)
\]

Fastify handles up to 30,000 req/sec with concurrent execution.

\subsection*{Guard \(\GuardH\) and Invariant \(\InvQ\)}

Guards enforce API contracts:
\begin{itemize}
\item \(\GuardH_{\text{schema}}\): Request body must validate against Zod schema
\item \(\GuardH_{\text{workflow}}\): Workflow must exist before case creation
\item \(\GuardH_{\text{workitem}}\): Work item must be enabled before start
\item \(\GuardH_{\text{state}}\): Task must be started before completion
\end{itemize}

Invariants:
\begin{itemize}
\item \(\InvQ_{\text{HATEOAS}}\): Links reflect actual workflow state
\item \(\InvQ_{\text{receipt}}\): Every state transition generates receipt
\item \(\InvQ_{\text{idempotent}}\): GET requests do not mutate state
\item \(\InvQ_{\text{REST}}\): HTTP methods follow REST semantics
\end{itemize}

HATEOAS correctness:
\[
\InvQ_{\text{links}}(\text{case}) \iff \forall \text{link} \in \text{enabledTasks}, \text{engine.isEnabled}(\text{link.workItemId})
\]

\subsection*{Provenance and Receipts}

API response with receipt:
\begin{lstlisting}[language=JavaScript]
{
  "case": {
    "id": "case-abc123",
    "workflowId": "approval",
    "status": "running",
    "data": { "amount": 1500 }
  },
  "receipt": {
    "timestamp": "2026-01-11T00:00:00.000Z",
    "operation": "completeTask",
    "caseId": "case-abc123",
    "taskId": "review",
    "actor": "manager@example.com",
    "hash": "blake3:def456...",
    "previousHash": "blake3:abc123..."
  },
  "_links": {
    "self": { "href": "/api/cases/case-abc123", "method": "GET" },
    "workflow": { "href": "/api/workflows/approval", "method": "GET" },
    "enabledTasks": [
      {
        "taskId": "approve",
        "workItemId": "wi-789",
        "actions": {
          "start": { "href": "/api/cases/case-abc123/tasks/wi-789/start", "method": "POST" }
        }
      }
    ]
  }
}
\end{lstlisting}

OpenAPI specification provides complete API provenance:
\begin{itemize}
\item All endpoints auto-documented
\item Request/response schemas from Zod
\item Example payloads generated
\item Available at \texttt{/docs/json}
\end{itemize}

\subsection*{Minimal Example}

\begin{lstlisting}[language=JavaScript]
import { createYAWLAPIServer } from '@unrdf/yawl-api';
import { createWorkflowEngine, SPLIT_TYPE, JOIN_TYPE } from '@unrdf/yawl';

const engine = createWorkflowEngine();

// Register workflow
engine.registerWorkflow({
  id: 'approval',
  name: 'Approval Workflow',
  tasks: [
    { id: 'submit', name: 'Submit', splitType: SPLIT_TYPE.AND, joinType: JOIN_TYPE.XOR },
    { id: 'review', name: 'Review', splitType: SPLIT_TYPE.XOR, joinType: JOIN_TYPE.XOR },
    { id: 'approve', name: 'Approve', splitType: SPLIT_TYPE.AND, joinType: JOIN_TYPE.XOR }
  ],
  flows: [
    { from: 'submit', to: 'review' },
    { from: 'review', to: 'approve' }
  ],
  startTaskId: 'submit',
  endTaskIds: ['approve']
});

// Create server
const server = await createYAWLAPIServer({
  engine,
  baseUrl: 'http://localhost:3000'
});

await server.listen({ port: 3000 });
console.log('API running at http://localhost:3000');
console.log('Swagger UI at http://localhost:3000/docs');
\end{lstlisting}

\subsection*{Open Questions}

\begin{enumerate}
\item Can HATEOAS link generation scale to workflows with 1000+ enabled tasks?
\item How to handle rate limiting for high-throughput case creation?
\item What is optimal caching strategy for workflow definitions?
\item Can WebSocket subscriptions enhance real-time state updates?
\end{enumerate}

% ============================================================================
% Package 33: @unrdf/react
% ============================================================================

\label{pkg:unrdf-react}
\section{\pkg{@unrdf/react} --- React Client Library}

\begin{pkgmeta}
Path & \texttt{packages/react} \\
Kind & js \\
Entrypoints & 5 files (main + AI semantic modules) \\
Dependencies & 4 (@unrdf/core, @unrdf/oxigraph, lru-cache, zod) \\
Blurb & AI semantic analysis tools for RDF knowledge graphs in React applications \\
\end{pkgmeta}

\subsection*{Observable \(\Oobs\) and Artifact \(\Aout\)}

Observables are RDF entities and semantic queries:

\[
\Oobs_{\text{react}} = \{ \text{entities}_{\text{RDF}}, \text{queries}_{\text{NL}}, \text{embeddings}, \text{patterns} \}
\]

Artifacts include semantic analysis results and recommendations:

\[
\Aout_{\text{react}} = \{ \text{similarities}, \text{anomalies}, \text{clusters}, \text{queries}_{\text{SPARQL}} \}
\]

The library observes:
\begin{itemize}
\item RDF entity descriptions for embedding generation
\item Natural language queries for SPARQL translation
\item Graph topology for anomaly detection
\item Usage patterns for query optimization
\end{itemize}

Produces:
\begin{itemize}
\item Semantic similarity scores between entities
\item NLP-to-SPARQL query translations
\item Anomaly detection reports
\item Entity recommendations based on graph structure
\end{itemize}

\subsection*{Type Signature \(\SigmaType\)}

Core module types from \texttt{packages/react/src/ai-semantic/}:

\begin{lstlisting}[language=JavaScript]
// Semantic analyzer signature
class SemanticAnalyzer {
  /**
   * @param {Store} store - RDF store
   * @param {Object} config
   * @param {number} config.embeddingDim - Vector dimension (default 384)
   * @param {boolean} config.cacheEnabled - Enable LRU cache
   */
  constructor(store, config = {}) { }

  /**
   * Compute semantic similarity between entities
   * @param {string} entityA - First entity URI
   * @param {string} entityB - Second entity URI
   * @returns {Promise<number>} Similarity score [0, 1]
   */
  async computeSimilarity(entityA, entityB) { }
}

// NLP query builder signature
class NLPQueryBuilder {
  /**
   * Translate natural language to SPARQL
   * @param {string} naturalLanguage - Query in natural language
   * @param {Object} options
   * @returns {Promise<string>} SPARQL query
   */
  async buildQuery(naturalLanguage, options) { }
}

// Anomaly detector signature
class AnomalyDetector {
  /**
   * Detect anomalies in RDF graph
   * @param {Object} config
   * @param {number} config.threshold - Anomaly score threshold
   * @returns {Promise<Array>} List of anomalies with scores
   */
  async detectAnomalies(config) { }
}
\end{lstlisting}

Type mapping:
\[
\SigmaType_{\text{similarity}}: (\text{URI}, \text{URI}) \to [0, 1]
\]
\[
\SigmaType_{\text{translate}}: \text{String}_{\text{NL}} \to \text{SPARQL}
\]

\subsection*{Reconciler \(\muRecon\)}

The reconciler maps natural language to structured queries:

\[
\muRecon_{\text{NLP}}: \Oobs_{\text{query}} \to \Aout_{\text{SPARQL}}
\]

NLP query pipeline:
\begin{enumerate}
\item Parse natural language query
\item Extract entities and predicates
\item Match to ontology concepts
\item Generate SPARQL patterns
\item Optimize query structure
\end{enumerate}

Semantic similarity reconciler:
\[
\muRecon_{\text{sim}}: (\text{Entity}_A, \text{Entity}_B) \to \text{Score} \in [0, 1]
\]

Implementation pattern:
\begin{lstlisting}[language=JavaScript]
async computeSimilarity(entityA, entityB) {
  // Retrieve entity embeddings (cached)
  const embeddingA = await this.getOrComputeEmbedding(entityA);
  const embeddingB = await this.getOrComputeEmbedding(entityB);

  // Cosine similarity
  const dotProduct = embeddingA.reduce((sum, val, i) =>
    sum + val * embeddingB[i], 0);
  const magnitudeA = Math.sqrt(embeddingA.reduce((sum, val) =>
    sum + val * val, 0));
  const magnitudeB = Math.sqrt(embeddingB.reduce((sum, val) =>
    sum + val * val, 0));

  return dotProduct / (magnitudeA * magnitudeB);
}
\end{lstlisting}

Anomaly detection reconciler:
\[
\muRecon_{\text{anomaly}}: \text{Graph} \to \{ \text{node}, \text{score}, \text{reason} \}^*
\]

\subsection*{Composition \(\PiMerge / \oplusMerge\)}

Analysis pipeline composition:
\[
\PiMerge(\text{embed}, \text{cluster}, \text{recommend}) = \text{SemanticEngine}
\]

Multi-stage analysis:
\begin{lstlisting}[language=JavaScript]
// Compose semantic analysis pipeline
const analyzer = new SemanticAnalyzer(store, { cacheEnabled: true });
const nlpBuilder = new NLPQueryBuilder(store);
const detector = new AnomalyDetector(store);

// Sequential composition
const query = await nlpBuilder.buildQuery("Find all research papers about AI");
const results = await store.query(query);
const anomalies = await detector.detectAnomalies({ threshold: 0.8 });
\end{lstlisting}

Parallel similarity computation:
\[
\Aout_{\text{similarities}} = \bigoplus_{(a, b) \in \text{Pairs}} \muRecon_{\text{sim}}(a, b)
\]

LRU cache composition reduces redundant computations.

\subsection*{Guard \(\GuardH\) and Invariant \(\InvQ\)}

Guards:
\begin{itemize}
\item \(\GuardH_{\text{entity}}\): Entity must exist in store before similarity
\item \(\GuardH_{\text{embedding}}\): Embedding dimension must be consistent (384)
\item \(\GuardH_{\text{query}}\): Natural language query must parse successfully
\item \(\GuardH_{\text{threshold}}\): Anomaly threshold must be in [0, 1]
\end{itemize}

Invariants:
\begin{itemize}
\item \(\InvQ_{\text{symmetric}}\): Similarity is symmetric: \(\text{sim}(a, b) = \text{sim}(b, a)\)
\item \(\InvQ_{\text{cache}}\): Cache returns identical results to fresh computation
\item \(\InvQ_{\text{normalized}}\): Similarity scores in [0, 1]
\item \(\InvQ_{\text{deterministic}}\): Same query produces same SPARQL
\end{itemize}

Cosine similarity invariant:
\[
\InvQ_{\text{cosine}}(u, v) = \frac{u \cdot v}{\|u\| \|v\|} \in [-1, 1]
\]

Cache consistency:
\[
\InvQ_{\text{cache}}(e) \iff \text{cached}(e) = \text{fresh}(e)
\]

\subsection*{Provenance and Receipts}

Semantic analysis receipt:
\begin{lstlisting}[language=JavaScript]
{
  operation: "semantic-similarity",
  timestamp: "2026-01-11T00:00:00.000Z",
  entityA: "http://example.org/Paper1",
  entityB: "http://example.org/Paper2",
  similarity: 0.87,
  embeddingDim: 384,
  cacheHit: true,
  computationTime: 12
}
\end{lstlisting}

NLP query translation receipt:
\begin{lstlisting}[language=JavaScript]
{
  naturalLanguage: "Find papers about machine learning",
  generatedSPARQL: "SELECT ?paper WHERE { ?paper rdf:type ex:Paper . ?paper ex:topic ex:MachineLearning }",
  entities: ["ex:Paper", "ex:MachineLearning"],
  predicates: ["rdf:type", "ex:topic"],
  confidence: 0.92
}
\end{lstlisting}

Provenance tracking:
\begin{itemize}
\item Embedding generation timestamps
\item Cache hit/miss statistics
\item Query translation confidence scores
\item Anomaly detection thresholds
\end{itemize}

\subsection*{Minimal Example}

\begin{lstlisting}[language=JavaScript]
import { SemanticAnalyzer, NLPQueryBuilder } from '@unrdf/react/ai-semantic';
import { createStore } from '@unrdf/oxigraph';

const store = createStore();

// Load RDF data
await store.load(rdfData, 'text/turtle');

// Initialize semantic analyzer
const analyzer = new SemanticAnalyzer(store, {
  embeddingDim: 384,
  cacheEnabled: true
});

// Compute similarity
const similarity = await analyzer.computeSimilarity(
  'http://example.org/Paper1',
  'http://example.org/Paper2'
);
console.log(`Similarity: ${similarity}`);

// NLP to SPARQL
const nlpBuilder = new NLPQueryBuilder(store);
const sparql = await nlpBuilder.buildQuery(
  "Find all papers published after 2020"
);
console.log(sparql);
\end{lstlisting}

\subsection*{Open Questions}

\begin{enumerate}
\item Can transformer-based embeddings improve similarity accuracy?
\item How to handle multilingual NLP queries with ontology alignment?
\item What is optimal cache size for embedding storage?
\item Can active learning improve anomaly detection thresholds?
\end{enumerate}

% ============================================================================
% Package 34: @unrdf/rdf-graphql
% ============================================================================

\label{pkg:unrdf-rdf-graphql}
\section{\pkg{@unrdf/rdf-graphql} --- GraphQL API Gateway}

\begin{pkgmeta}
Path & \texttt{packages/rdf-graphql} \\
Kind & js \\
Entrypoints & 4 files (adapter, schema, query, resolver) \\
Dependencies & 4 (graphql, @graphql-tools/schema, @unrdf/oxigraph, zod) \\
Blurb & Type-safe GraphQL interface with automatic schema generation from RDF ontologies \\
\end{pkgmeta}

\subsection*{Observable \(\Oobs\) and Artifact \(\Aout\)}

Observables are RDF ontologies and GraphQL queries:

\[
\Oobs_{\text{graphql}} = \{ \text{ontology}_{\text{RDFS/OWL}}, \text{queries}_{\text{GraphQL}}, \text{instances}_{\text{RDF}} \}
\]

Artifacts include GraphQL schemas and query results:

\[
\Aout_{\text{graphql}} = \{ \text{Schema}_{\text{SDL}}, \text{Resolvers}, \text{QueryResults}, \text{Cache} \}
\]

The adapter observes:
\begin{itemize}
\item RDFS/OWL class definitions via SPARQL introspection
\item Property domains and ranges for field generation
\item GraphQL queries from clients
\item RDF instance data for resolver execution
\end{itemize}

Produces:
\begin{itemize}
\item GraphQL Schema Definition Language (SDL)
\item Resolver functions bound to SPARQL queries
\item Type-safe query execution results
\item Query result cache with statistics
\end{itemize}

\subsection*{Type Signature \(\SigmaType\)}

Adapter configuration from \texttt{adapter.mjs}:

\begin{lstlisting}[language=JavaScript]
const AdapterConfigSchema = z.object({
  namespaces: z.record(z.string()).optional(),
  excludeClasses: z.array(z.string()).optional(),
  includeInferred: z.boolean().optional(),
  enableCache: z.boolean().optional(),
  typeMapping: z.record(z.string()).optional()
});

class RDFGraphQLAdapter {
  constructor(config = {}) {
    this.config = AdapterConfigSchema.parse(config);
    this.store = createStore();
    this.schema = null;
    this.schemaGenerator = new RDFSchemaGenerator(config);
    this.resolverFactory = null;
  }

  async loadOntology(rdfData, format, baseIRI) { }
  async loadData(rdfData, format, baseIRI) { }
  generateSchema(options) { }
  async executeQuery(query, variables, context) { }
}
\end{lstlisting}

Type mapping rules:
\[
\SigmaType_{\text{RDF}}(\text{rdfs:Class}) \mapsto \SigmaType_{\text{GraphQL}}(\text{ObjectType})
\]
\[
\SigmaType_{\text{XSD}}(\text{xsd:string}) \mapsto \SigmaType_{\text{GraphQL}}(\text{String})
\]

Complete mapping:
\begin{itemize}
\item \texttt{owl:Class} $\to$ GraphQL Object Type
\item \texttt{owl:DatatypeProperty} $\to$ GraphQL Scalar Field
\item \texttt{owl:ObjectProperty} $\to$ GraphQL Object Field
\item \texttt{xsd:integer} $\to$ GraphQL Int
\item \texttt{xsd:decimal} $\to$ GraphQL Float
\item \texttt{xsd:boolean} $\to$ GraphQL Boolean
\end{itemize}

\subsection*{Reconciler \(\muRecon\)}

The reconciler generates GraphQL schemas from RDF ontologies:

\[
\muRecon_{\text{schema}}: \Oobs_{\text{ontology}} \to \Aout_{\text{GraphQLSchema}}
\]

Schema generation pipeline from \texttt{schema-generator.mjs}:
\begin{enumerate}
\item \texttt{loadOntology()}: Parse RDF into internal model
\item \texttt{introspectClasses()}: SPARQL query for all \texttt{rdfs:Class}
\item \texttt{introspectProperties()}: SPARQL query for properties with domain/range
\item \texttt{generateSchema()}: Traverse ontology, emit GraphQL types
\item \texttt{createResolvers()}: Bind resolvers to SPARQL backend
\end{enumerate}

Query reconciler from \texttt{query-builder.mjs}:
\[
\muRecon_{\text{query}}: \text{GraphQL Query} \to \text{SPARQL Query} \to \text{Results}
\]

SPARQL translation algorithm:
\begin{lstlisting}[language=JavaScript]
class SPARQLQueryBuilder {
  buildListQuery(info, typeIRI, args) {
    const { limit = 10, offset = 0 } = args;

    let sparql = `
      SELECT ?subject ?predicate ?object WHERE {
        ?subject rdf:type <${typeIRI}> .
        ?subject ?predicate ?object .
      }
      LIMIT ${limit}
      OFFSET ${offset}
    `;

    return sparql;
  }

  buildSingleQuery(info, typeIRI, id) {
    return `
      SELECT ?predicate ?object WHERE {
        <${id}> rdf:type <${typeIRI}> .
        <${id}> ?predicate ?object .
      }
    `;
  }
}
\end{lstlisting}

Resolver factory from \texttt{resolver.mjs}:
\begin{lstlisting}[language=JavaScript]
class RDFResolverFactory {
  createResolvers(schema, store, queryBuilder) {
    const resolvers = {
      Query: {}
    };

    for (const [typeName, typeIRI] of schema.types) {
      // List query resolver
      resolvers.Query[`${typeName}s`] = async (parent, args, context, info) => {
        const sparql = queryBuilder.buildListQuery(info, typeIRI, args);
        const results = await store.query(sparql);
        return this.formatResults(results);
      };

      // Single item resolver
      resolvers.Query[typeName] = async (parent, args, context, info) => {
        const sparql = queryBuilder.buildSingleQuery(info, typeIRI, args.id);
        const results = await store.query(sparql);
        return this.formatSingleResult(results);
      };
    }

    return resolvers;
  }
}
\end{lstlisting}

\subsection*{Composition \(\PiMerge / \oplusMerge\)}

Adapter lifecycle composition:
\[
\PiMerge(\text{load}, \text{generate}, \text{execute}) = \text{Adapter}
\]

Method chaining:
\begin{lstlisting}[language=JavaScript]
const adapter = new RDFGraphQLAdapter({ enableCache: true });

await adapter
  .loadOntology(ontologyRDF)
  .then(() => adapter.generateSchema())
  .then(() => adapter.executeQuery(query));
\end{lstlisting}

Multiple ontologies merged via union:
\[
\Aout_{\text{schema}} = \bigoplus_{i} \text{loadOntology}(\text{onto}_i)
\]

Schema merging preserves type safety when no conflicts exist.

\subsection*{Guard \(\GuardH\) and Invariant \(\InvQ\)}

Guards enforce preconditions:
\begin{itemize}
\item \(\GuardH_{\text{schema}}\): Schema must be generated before query execution
\item \(\GuardH_{\text{ontology}}\): Ontology must be loaded before schema generation
\item \(\GuardH_{\text{validation}}\): Config must validate against \texttt{AdapterConfigSchema}
\item \(\GuardH_{\text{type}}\): GraphQL query types must exist in schema
\end{itemize}

Invariants:
\begin{itemize}
\item \(\InvQ_{\text{type}}\): Generated schema respects RDF type constraints
\item \(\InvQ_{\text{cache}}\): Cache returns equivalent results to fresh query
\item \(\InvQ_{\text{introspection}}\): Introspection matches loaded ontology
\item \(\InvQ_{\text{resolver}}\): Resolvers produce valid GraphQL responses
\end{itemize}

Cache consistency:
\[
\InvQ_{\text{cache}}(\text{query}, t) \iff \text{cached}(\text{query}) = \text{fresh}(\text{query})
\]

Type safety:
\[
\InvQ_{\text{type}}(\text{field}) \iff \text{GraphQLType}(\text{field}) \equiv \text{RDFRange}(\text{property})
\]

\subsection*{Provenance and Receipts}

Query execution trace:
\begin{lstlisting}[language=JavaScript]
{
  graphqlQuery: "{ Person { id name age } }",
  generatedSPARQL: "SELECT ?subject ?name ?age WHERE { ?subject rdf:type ex:Person . ?subject ex:name ?name . ?subject ex:age ?age }",
  executionTime: 45,
  resultCount: 10,
  cacheHit: false,
  timestamp: "2026-01-11T00:00:00.000Z"
}
\end{lstlisting}

Statistics API from \texttt{adapter.mjs}:
\begin{lstlisting}[language=JavaScript]
const stats = await adapter.getStatistics();
// {
//   tripleCount: 1000,
//   classCount: 50,
//   propertyCount: 120,
//   instanceCount: 200
// }

const cacheStats = adapter.getCacheStats();
// { enabled: true, size: 42 }
\end{lstlisting}

Introspection provides ontology provenance:
\begin{lstlisting}[language=JavaScript]
const classes = adapter.introspectClasses();
// [{ iri: "ex:Person", label: "Person", comment: "..." }, ...]

const properties = adapter.introspectProperties();
// [{ iri: "ex:name", domain: "ex:Person", range: "xsd:string" }, ...]
\end{lstlisting}

\subsection*{Minimal Example}

\begin{lstlisting}[language=JavaScript]
import { createAdapter } from '@unrdf/rdf-graphql';

const adapter = createAdapter({
  namespaces: { ex: 'http://example.org/' },
  enableCache: true
});

// Load ontology
await adapter.loadOntology(`
  @prefix ex: <http://example.org/> .
  @prefix rdfs: <http://www.w3.org/2000/01/rdf-schema#> .
  @prefix xsd: <http://www.w3.org/2001/XMLSchema#> .

  ex:Person a rdfs:Class ;
    rdfs:label "Person" .

  ex:name a rdf:Property ;
    rdfs:domain ex:Person ;
    rdfs:range xsd:string .

  ex:age a rdf:Property ;
    rdfs:domain ex:Person ;
    rdfs:range xsd:integer .
`);

// Load instance data
await adapter.loadData(`
  @prefix ex: <http://example.org/> .

  <http://example.org/people/alice> a ex:Person ;
    ex:name "Alice Smith" ;
    ex:age 30 .
`);

// Generate schema
adapter.generateSchema();

// Execute GraphQL query
const result = await adapter.executeQuery(`
  query {
    Person(id: "http://example.org/people/alice") {
      id
      name
      age
    }
  }
`);

console.log(result.data.Person);
// { id: "http://example.org/people/alice", name: "Alice Smith", age: 30 }
\end{lstlisting}

\subsection*{Open Questions}

\begin{enumerate}
\item Can nested GraphQL queries map efficiently to SPARQL property paths?
\item How to handle GraphQL mutations with transactional RDF updates?
\item What is cache invalidation strategy for dynamic ontologies?
\item Can federation support distributed RDF data sources?
\end{enumerate}

% ============================================================================
% Package 35: @unrdf/yawl-realtime
% ============================================================================

\label{pkg:unrdf-yawl-realtime}
\section{\pkg{@unrdf/yawl-realtime} --- WebSocket Real-Time Updates}

\begin{pkgmeta}
Path & \texttt{packages/yawl-realtime} \\
Kind & js \\
Entrypoints & 3 files (index, server, client) \\
Dependencies & 3 (socket.io, socket.io-client, @unrdf/yawl, zod) \\
Blurb & Real-time collaboration framework using Socket.io with optimistic locking and CRDT-inspired merging \\
\end{pkgmeta}

\subsection*{Observable \(\Oobs\) and Artifact \(\Aout\)}

Observables are real-time events and distributed state:

\[
\Oobs_{\text{realtime}} = \{ \text{events}_{\text{YAWL}}, \text{connections}, \text{locks}, \text{timestamps}_{\text{Lamport}} \}
\]

Artifacts include synchronized state and conflict resolutions:

\[
\Aout_{\text{realtime}} = \{ \text{state}_{\text{synced}}, \text{receipts}, \text{locks}_{\text{active}}, \text{resolutions} \}
\]

The framework observes:
\begin{itemize}
\item YAWL engine events (task enabled, started, completed)
\item WebSocket connection state changes
\item Task claim attempts with Lamport timestamps
\item State updates from multiple clients
\end{itemize}

Produces:
\begin{itemize}
\item Real-time event broadcasts to connected clients
\item Optimistic lock acquisition/release notifications
\item CRDT-inspired state merges (LWW for data, Add-Wins for work items)
\item Conflict resolution reports with causality tracking
\end{itemize}

\subsection*{Type Signature \(\SigmaType\)}

Server configuration:

\begin{lstlisting}[language=JavaScript]
class YAWLRealtimeServer {
  constructor(engine, options = {}) {
    this.engine = engine;
    this.io = new Server(options.port, options.corsOptions);
    this.lockManager = new OptimisticLockManager();
    this.stateSync = new StateSyncManager();
  }

  async start() { }
  async stop() { }
  getStats() { }
}
\end{lstlisting}

Client configuration:

\begin{lstlisting}[language=JavaScript]
class YAWLRealtimeClient {
  constructor(options) {
    this.serverUrl = options.serverUrl;
    this.userId = options.userId;
    this.socket = null;
    this.lamportClock = 0;
  }

  async connect() { }
  async disconnect() { }
  async claimTask(caseId, workItemId, options) { }
  async completeTask(caseId, workItemId, output) { }
  async releaseTask(caseId, workItemId) { }
  async syncState(caseId) { }
}
\end{lstlisting}

Lock manager type:
\[
\SigmaType_{\text{lock}}: (\text{WorkItemId}, \text{UserId}, \text{Timestamp}) \to \{ \text{success}, \text{conflict}? \}
\]

State sync type:
\[
\SigmaType_{\text{sync}}: \text{StateUpdate} \to \text{MergedState}
\]

\subsection*{Reconciler \(\muRecon\)}

The reconciler maintains distributed consistency:

\[
\muRecon_{\text{realtime}}: \Oobs_{\text{events}} \to \Aout_{\text{synced}}
\]

Optimistic locking algorithm:
\begin{lstlisting}[language=JavaScript]
class OptimisticLockManager {
  acquire(workItemId, caseId, userId, timestamp) {
    const existingLock = this.locks.get(workItemId);

    if (!existingLock) {
      // No conflict, acquire immediately
      this.locks.set(workItemId, {
        workItemId,
        caseId,
        userId,
        timestamp,
        acquiredAt: Date.now()
      });
      return { success: true };
    }

    // Lamport timestamp conflict resolution
    if (timestamp > existingLock.timestamp) {
      // Higher timestamp wins
      const oldUserId = existingLock.userId;
      this.locks.set(workItemId, {
        workItemId,
        caseId,
        userId,
        timestamp,
        acquiredAt: Date.now()
      });
      return {
        success: true,
        conflict: {
          type: 'timestamp',
          resolution: 'won',
          previousOwner: oldUserId
        }
      };
    }

    // Conflict, existing lock wins
    return {
      success: false,
      conflict: {
        type: 'timestamp',
        resolution: 'lost',
        currentOwner: existingLock.userId,
        timestamp: existingLock.timestamp
      }
    };
  }

  release(workItemId, userId) {
    const lock = this.locks.get(workItemId);
    if (lock && lock.userId === userId) {
      this.locks.delete(workItemId);
      return { success: true };
    }
    return { success: false, error: 'Not lock owner' };
  }
}
\end{lstlisting}

CRDT-inspired state merge:
\begin{lstlisting}[language=JavaScript]
class StateSyncManager {
  mergeState(caseId, update, receiptHash) {
    const currentState = this.states.get(caseId) || {
      data: {},
      workItems: new Map(),
      receiptChain: []
    };

    // Last-Write-Wins for data
    const mergedData = { ...currentState.data, ...update.data };

    // Add-Wins for work items (set union)
    const mergedWorkItems = new Map([
      ...currentState.workItems,
      ...update.workItems
    ]);

    // Append receipt to chain
    const mergedReceipts = [
      ...currentState.receiptChain,
      { hash: receiptHash, timestamp: Date.now() }
    ];

    const mergedState = {
      data: mergedData,
      workItems: mergedWorkItems,
      receiptChain: mergedReceipts
    };

    this.states.set(caseId, mergedState);
    return { success: true, state: mergedState };
  }

  verifyReceiptChain(caseId, expectedHash) {
    const state = this.states.get(caseId);
    if (!state || !state.receiptChain.length) {
      return { valid: false, error: 'No receipt chain' };
    }

    const latestReceipt = state.receiptChain[state.receiptChain.length - 1];
    return {
      valid: latestReceipt.hash === expectedHash,
      latestHash: latestReceipt.hash
    };
  }
}
\end{lstlisting}

\subsection*{Composition \(\PiMerge / \oplusMerge\)}

Client-server composition:
\[
\PiMerge(\text{Connect}, \text{Claim}, \text{Execute}, \text{Sync}) = \text{CollaborativeSession}
\]

Event broadcast composition:
\[
\Aout_{\text{broadcasts}} = \bigoplus_{c \in \text{Clients}} \text{emit}(c, \text{event})
\]

Multi-client workflow example:
\begin{lstlisting}[language=JavaScript]
// Server broadcasts to all connected clients
io.on('connection', (socket) => {
  engine.on('task:enabled', (event) => {
    // Broadcast to all clients
    io.emit('yawl:event', event);
  });

  socket.on('task:claim', async (data) => {
    const result = lockManager.acquire(
      data.workItemId,
      data.caseId,
      socket.userId,
      data.timestamp
    );

    if (result.success) {
      // Broadcast lock to all clients
      io.emit('task:locked', {
        workItemId: data.workItemId,
        userId: socket.userId
      });
    }

    socket.emit('task:claimed', result);
  });
});
\end{lstlisting}

\subsection*{Guard \(\GuardH\) and Invariant \(\InvQ\)}

Guards:
\begin{itemize}
\item \(\GuardH_{\text{connected}}\): Client must connect before operations
\item \(\GuardH_{\text{ownership}}\): Only lock owner can complete task
\item \(\GuardH_{\text{timestamp}}\): Lamport clock must monotonically increase
\item \(\GuardH_{\text{receipt}}\): Receipt hash must match for conflict-free merge
\end{itemize}

Invariants:
\begin{itemize}
\item \(\InvQ_{\text{lock}}\): At most one lock per work item
\item \(\InvQ_{\text{causal}}\): Receipt chain maintains causal ordering
\item \(\InvQ_{\text{LWW}}\): Last-Write-Wins produces consistent data state
\item \(\InvQ_{\text{convergence}}\): All clients converge to same state given all updates
\end{itemize}

Lock uniqueness:
\[
\InvQ_{\text{lock}}(w) \iff |\{ \text{lock} \mid \text{lock.workItemId} = w \}| \leq 1
\]

Causal consistency:
\[
\InvQ_{\text{causal}}(\text{chain}) \iff \forall i < j, \text{chain}[i].\text{timestamp} < \text{chain}[j].\text{timestamp}
\]

Convergence theorem (CRDT-inspired):
\[
\InvQ_{\text{conv}}(S_1, S_2) \iff \text{applyAll}(\text{updates}) \Rightarrow S_1 \equiv S_2
\]

\subsection*{Provenance and Receipts}

Collaborative session receipt:
\begin{lstlisting}[language=JavaScript]
{
  sessionId: "session-abc123",
  caseId: "case-456",
  participants: [
    { userId: "alice@example.com", joinedAt: "2026-01-11T00:00:00.000Z" },
    { userId: "bob@example.com", joinedAt: "2026-01-11T00:01:00.000Z" }
  ],
  lockEvents: [
    {
      workItemId: "wi-789",
      userId: "alice@example.com",
      action: "acquire",
      timestamp: 100,
      lamportClock: 5
    },
    {
      workItemId: "wi-789",
      userId: "alice@example.com",
      action: "release",
      timestamp: 150,
      lamportClock: 8
    }
  ],
  stateUpdates: [
    {
      userId: "alice@example.com",
      operation: "completeTask",
      receiptHash: "blake3:abc...",
      timestamp: 145
    }
  ],
  conflicts: [
    {
      workItemId: "wi-789",
      conflictType: "concurrent-claim",
      winner: "alice@example.com",
      loser: "bob@example.com",
      resolution: "lamport-timestamp",
      timestamp: 102
    }
  ]
}
\end{lstlisting}

Provenance tracking:
\begin{itemize}
\item Lamport clock values for causality
\item Lock acquisition/release events
\item State merge operations with hashes
\item Conflict resolutions with reasoning
\end{itemize}

\subsection*{Minimal Example}

\begin{lstlisting}[language=JavaScript]
import { createWorkflowEngine } from '@unrdf/yawl';
import { YAWLRealtimeServer, YAWLRealtimeClient } from '@unrdf/yawl-realtime';

// Server setup
const engine = createWorkflowEngine();
const server = new YAWLRealtimeServer(engine, {
  port: 3000,
  corsOptions: { origin: '*' }
});
await server.start();

// Client A
const clientA = new YAWLRealtimeClient({
  serverUrl: 'http://localhost:3000',
  userId: 'alice@example.com'
});
await clientA.connect();

// Client B
const clientB = new YAWLRealtimeClient({
  serverUrl: 'http://localhost:3000',
  userId: 'bob@example.com'
});
await clientB.connect();

// Alice claims task
const resultA = await clientA.claimTask('case-123', 'wi-456');
console.log(resultA.success); // true

// Bob attempts to claim same task (concurrent)
const resultB = await clientB.claimTask('case-123', 'wi-456');
console.log(resultB.success); // false (conflict)
console.log(resultB.conflict.currentOwner); // alice@example.com

// Alice completes task
await clientA.completeTask('case-123', 'wi-456', {
  decision: 'approved'
});

// Both clients receive completion event
clientA.on('task:completed', (event) => {
  console.log('Task completed:', event);
});
\end{lstlisting}

\subsection*{Open Questions}

\begin{enumerate}
\item Can distributed locks scale to 1000+ concurrent users per workflow?
\item How to handle network partitions with split-brain scenarios?
\item What is optimal Lamport clock synchronization frequency?
\item Can vector clocks improve causality tracking for complex workflows?
\end{enumerate}

% ============================================================================
% Package 36: @unrdf/serverless
% ============================================================================

\label{pkg:unrdf-serverless}
\section{\pkg{@unrdf/serverless} --- AWS Serverless Deployment}

\begin{pkgmeta}
Path & \texttt{packages/serverless} \\
Kind & js \\
Entrypoints & 5 files (index, cdk, deploy, api, storage) \\
Dependencies & 10 (aws-cdk-lib, constructs, esbuild, zod, @unrdf/core, @unrdf/oxigraph) \\
Blurb & One-click AWS deployment with Lambda, API Gateway, DynamoDB, and CloudFront CDN \\
\end{pkgmeta}

\subsection*{Observable \(\Oobs\) and Artifact \(\Aout\)}

Observables are application code and infrastructure requirements:

\[
\Oobs_{\text{deploy}} = \{ \text{handlers}_{\text{Lambda}}, \text{config}_{\text{CDK}}, \text{deps}, \text{env} \}
\]

Artifacts include deployed AWS resources:

\[
\Aout_{\text{infra}} = \{ \text{Lambdas}, \text{API Gateway}, \text{DynamoDB}, \text{CloudFront}, \text{CloudFormation} \}
\]

The toolkit observes:
\begin{itemize}
\item Lambda handler code (MJS files)
\item CDK stack definitions in TypeScript/JavaScript
\item API endpoint configurations
\item Environment variables and secrets
\end{itemize}

Produces:
\begin{itemize}
\item Optimized Lambda bundles via esbuild
\item CloudFormation templates
\item Deployed API Gateway endpoints with CORS
\item DynamoDB tables for RDF triple storage
\item CloudFront CDN distributions (optional)
\end{itemize}

\subsection*{Type Signature \(\SigmaType\)}

Stack configuration schema:

\begin{lstlisting}[language=JavaScript]
const StackConfigSchema = z.object({
  stackName: z.string(),
  environment: z.enum(['dev', 'staging', 'prod']),
  region: z.string(),
  memorySizeMb: z.number().min(128).max(10240),
  timeoutSeconds: z.number().min(1).max(900),
  enableCdn: z.boolean().default(false),
  enableStreaming: z.boolean().default(false),
  tableName: z.string().optional(),
  apiName: z.string().optional()
});
\end{lstlisting}

CDK stack signature:
\[
\SigmaType_{\text{stack}}: \text{Config} \to \text{CDK.Stack} \to \text{AWS Resources}
\]

Lambda bundler signature from \texttt{deploy/bundler.mjs}:

\begin{lstlisting}[language=JavaScript]
class LambdaBundler {
  constructor(options) {
    this.entryPoint = options.entryPoint;
    this.outDir = options.outDir;
    this.minify = options.minify ?? true;
  }

  async bundle() {
    const result = await esbuild.build({
      entryPoints: [this.entryPoint],
      bundle: true,
      platform: 'node',
      target: 'node18',
      format: 'esm',
      outdir: this.outDir,
      minify: this.minify,
      external: ['@aws-sdk/*'], // Provided by Lambda runtime
      metafile: true,
      sourcemap: true
    });

    return {
      sizeBytes: result.metafile.outputs[0].bytes,
      warnings: result.warnings
    };
  }

  static async analyzeBundleSize(metafilePath) {
    // Bundle size analysis from metafile
  }
}
\end{lstlisting}

DynamoDB adapter type:
\[
\SigmaType_{\text{storage}}: \text{RDF Quad} \to \text{DynamoDB Item} \to \text{RDF Quad}
\]

\subsection*{Reconciler \(\muRecon\)}

The reconciler transforms application code to AWS infrastructure:

\[
\muRecon_{\text{deploy}}: \Oobs_{\text{app}} \to \Aout_{\text{AWS}}
\]

Deployment pipeline:
\begin{enumerate}
\item \texttt{LambdaBundler.bundle()}: MJS $\to$ Optimized bundle
\item \texttt{UNRDFStack.addLambda()}: Bundle $\to$ CDK Lambda construct
\item \texttt{ApiGatewayConfig.addEndpoint()}: Handler $\to$ API route
\item \texttt{DynamoDBAdapter.createTable()}: Schema $\to$ DynamoDB table
\item \texttt{cdk deploy}: CDK $\to$ CloudFormation $\to$ AWS resources
\end{enumerate}

CDK stack creation from \texttt{cdk/index.mjs}:
\begin{lstlisting}[language=JavaScript]
export function createUNRDFStack(app, id, config) {
  const stack = new Stack(app, id, {
    env: { region: config.region }
  });

  // DynamoDB table for triples
  const triplesTable = new Table(stack, 'TriplesTable', {
    partitionKey: { name: 'subject', type: AttributeType.STRING },
    sortKey: { name: 'predicateObject', type: AttributeType.STRING },
    billingMode: BillingMode.PAY_PER_REQUEST,
    pointInTimeRecovery: true
  });

  // Global secondary indexes for query patterns
  triplesTable.addGlobalSecondaryIndex({
    indexName: 'PredicateIndex',
    partitionKey: { name: 'predicate', type: AttributeType.STRING },
    sortKey: { name: 'subjectObject', type: AttributeType.STRING }
  });

  triplesTable.addGlobalSecondaryIndex({
    indexName: 'ObjectIndex',
    partitionKey: { name: 'object', type: AttributeType.STRING },
    sortKey: { name: 'subjectPredicate', type: AttributeType.STRING }
  });

  // Lambda function
  const queryHandler = new Function(stack, 'QueryHandler', {
    runtime: Runtime.NODEJS_18_X,
    memorySize: config.memorySizeMb,
    timeout: Duration.seconds(config.timeoutSeconds),
    handler: 'index.handler',
    code: Code.fromAsset(config.bundlePath),
    environment: {
      TRIPLES_TABLE: triplesTable.tableName,
      ENVIRONMENT: config.environment
    },
    tracing: Tracing.ACTIVE // X-Ray
  });

  triplesTable.grantReadWriteData(queryHandler);

  // API Gateway
  const api = new RestApi(stack, 'UnrdfApi', {
    restApiName: config.apiName || 'unrdf-api',
    deployOptions: {
      stageName: config.environment
    }
  });

  const integration = new LambdaIntegration(queryHandler);
  api.root.addResource('query').addMethod('POST', integration);

  // Optional CloudFront CDN
  if (config.enableCdn) {
    new Distribution(stack, 'CDN', {
      defaultBehavior: {
        origin: new HttpOrigin(api.url),
        cachePolicy: CachePolicy.CACHING_OPTIMIZED
      }
    });
  }

  return stack;
}
\end{lstlisting}

Runtime request reconciler:
\[
\muRecon_{\text{request}}: \text{HTTP Request} \to \text{Lambda} \to \text{DynamoDB} \to \text{HTTP Response}
\]

\subsection*{Composition \(\PiMerge / \oplusMerge\)}

CDK stack composition:
\[
\PiMerge(\text{Lambdas}, \text{API}, \text{Storage}, \text{CDN}) = \text{UNRDFStack}
\]

Sequential deployment stages:
\begin{lstlisting}[language=JavaScript]
import { App } from 'aws-cdk-lib';
import { createUNRDFStack, LambdaBundler } from '@unrdf/serverless';

const app = new App();

// Bundle Lambda functions
const bundler = new LambdaBundler({
  entryPoint: './src/handler.mjs',
  outDir: './dist/lambda',
  minify: true
});
const bundleMetadata = await bundler.bundle();

// Create stack with bundled functions
const stack = createUNRDFStack(app, 'UnrdfProdStack', {
  environment: 'prod',
  region: 'us-east-1',
  memorySizeMb: 2048,
  timeoutSeconds: 30,
  enableCdn: true,
  bundlePath: './dist/lambda'
});

app.synth();
\end{lstlisting}

Multi-region deployment:
\[
\Aout_{\text{global}} = \bigoplus_{r \in \text{Regions}} \text{deploy}(r)
\]

Parallel stack deployment across regions for global distribution.

\subsection*{Guard \(\GuardH\) and Invariant \(\InvQ\)}

Guards enforce AWS limits:
\begin{itemize}
\item \(\GuardH_{\text{memory}}\): Lambda memory 128MB--10GB
\item \(\GuardH_{\text{timeout}}\): Lambda timeout 1--900 seconds
\item \(\GuardH_{\text{bundle}}\): Bundle size $<$ 250MB uncompressed, $<$ 50MB zipped
\item \(\GuardH_{\text{region}}\): Region must be valid AWS region
\item \(\GuardH_{\text{table}}\): DynamoDB table name must be unique per region
\end{itemize}

Invariants:
\begin{itemize}
\item \(\InvQ_{\text{idempotent}}\): Re-deploying same code produces equivalent stack
\item \(\InvQ_{\text{storage}}\): DynamoDB round-trip preserves RDF quad structure
\item \(\InvQ_{\text{api}}\): All endpoints return valid HTTP status codes
\item \(\InvQ_{\text{cost}}\): Pay-per-request billing ensures no idle costs
\end{itemize}

RDF round-trip invariant:
\[
\InvQ_{\text{roundtrip}}(q) \iff \text{fromDynamoDB}(\text{toDynamoDB}(q)) \equiv q
\]

Deployment idempotence:
\[
\InvQ_{\text{deploy}}(c) \iff \text{deploy}(c)_1 \equiv \text{deploy}(c)_2
\]

\subsection*{Provenance and Receipts}

Deployment receipt from CloudFormation:
\begin{lstlisting}[language=JavaScript]
{
  stackName: "UnrdfProdStack",
  region: "us-east-1",
  deploymentTime: "2026-01-11T00:00:00.000Z",
  resources: {
    lambdas: ["QueryHandler-AbC123"],
    apis: ["https://api123.execute-api.us-east-1.amazonaws.com/prod"],
    tables: ["TriplesTable-prod"],
    cdns: ["d1234567890.cloudfront.net"]
  },
  bundleHashes: {
    "queryHandler": "sha256:abc123...",
    "updateHandler": "sha256:def456..."
  },
  stackId: "arn:aws:cloudformation:us-east-1:123456789012:stack/UnrdfProdStack/...",
  stackStatus: "CREATE_COMPLETE"
}
\end{lstlisting}

Bundle analysis receipt:
\begin{lstlisting}[language=JavaScript]
{
  entryPoint: "./src/handler.mjs",
  outputPath: "./dist/lambda/index.js",
  sizeBytes: 45678,
  sizeCompressed: 12345,
  dependencies: {
    "@unrdf/core": { bytes: 25000 },
    "@unrdf/oxigraph": { bytes: 15000 },
    "zod": { bytes: 5000 }
  },
  treeshakeEfficiency: 0.72,
  buildTime: 450
}
\end{lstlisting}

Provenance tracking:
\begin{itemize}
\item CloudFormation stack events with timestamps
\item Bundle size and dependency analysis
\item Lambda execution metrics via X-Ray
\item DynamoDB capacity and performance metrics
\end{itemize}

\subsection*{Minimal Example}

\begin{lstlisting}[language=JavaScript]
// 1. Define Lambda handler (src/handler.mjs)
import { createAdapterFromEnv, createApiResponse } from '@unrdf/serverless';

export async function handler(event) {
  try {
    const adapter = createAdapterFromEnv();
    const body = JSON.parse(event.body);

    const results = await adapter.queryTriples({
      subject: body.subject,
      predicate: body.predicate
    });

    return createApiResponse(200, {
      results,
      count: results.length
    });
  } catch (error) {
    return createErrorResponse(error, 500);
  }
}

// 2. Create CDK stack (cdk.mjs)
import { App } from 'aws-cdk-lib';
import { createUNRDFStack, LambdaBundler } from '@unrdf/serverless';

const app = new App();

// Bundle Lambda
const bundler = new LambdaBundler({
  entryPoint: './src/handler.mjs',
  outDir: './dist'
});
await bundler.bundle();

// Create stack
const stack = createUNRDFStack(app, 'MyRDFApp', {
  environment: 'prod',
  region: 'us-east-1',
  memorySizeMb: 2048,
  timeoutSeconds: 30,
  enableCdn: true,
  bundlePath: './dist'
});

app.synth();

// 3. Deploy
// $ cdk deploy
\end{lstlisting}

\subsection*{Open Questions}

\begin{enumerate}
\item Can Lambda cold starts be mitigated for large RDF graphs with provisioned concurrency?
\item How to optimize SPARQL query execution in DynamoDB with limited query patterns?
\item What is cost-optimal sharding strategy for multi-billion triple datasets?
\item Can Step Functions orchestrate complex RDF processing pipelines?
\end{enumerate}

% ============================================================================
% End of Agent 8 Packages
% ============================================================================
