% agent_8_packages.tex
% Agent 8 Package Documentation: Packages 35-41
% Generated: 2025-12-27

% ============================================================================
% Package 35: unrdf-project-engine
% ============================================================================

\label{pkg:unrdf-project-engine}
\section{\pkg{@unrdf/project-engine} --- UNRDF Project Engine}

\begin{pkgmeta}
Path & \texttt{packages/project-engine} \\
Kind & js \\
Entrypoints & 1 file \\
Dependencies & 3 \\
Blurb & Self-hosting Tools and Infrastructure (Development Only) \\
\end{pkgmeta}

\subsection*{Observable \(\Oobs\) and Artifact \(\Aout\)}

The project engine operates on filesystem structures as observables:

\[
\Oobs_{\text{fs}} = \{ \text{paths}, \text{contents}, \text{timestamps} \}
\]

Artifacts include RDF-encoded project models, domain inferences, and hook policies:

\[
\Aout_{\text{model}} = \store \cup \text{PolicyPacks} \cup \text{Reports}
\]

Primary observables:
\begin{itemize}
\item File tree structure via \texttt{scanFileSystemToStore}
\item Stack signatures (package.json, tsconfig, etc.)
\item Code complexity metrics from AST traversal
\item Drift between baseline and current structure
\end{itemize}

Artifacts produced:
\begin{itemize}
\item RDF graphs representing project structure
\item Knowledge hooks derived from pattern analysis
\item MAPEK loop execution reports
\item Test skeletons and API contract schemas
\end{itemize}

\subsection*{Type Signature \(\SigmaType\)}

All operations validated via Zod schemas:

\begin{lstlisting}[language=JavaScript]
const ProjectEngineConfigSchema = z.object({
  rootPath: z.string(),
  includePatterns: z.array(z.string()).optional(),
  excludePatterns: z.array(z.string()).optional(),
  enableHotspotAnalysis: z.boolean().default(true),
  enableGapDetection: z.boolean().default(true),
});
\end{lstlisting}

Key type signatures from exported modules:
\begin{itemize}
\item \texttt{FieldInfoSchema}: Field-level type metadata
\item \texttt{AuditResultSchema}: Type consistency audit results
\item \texttt{ValidationResultSchema}: API contract validation
\item \texttt{DriftEntrySchema}: Documentation drift detection
\end{itemize}

Type preservation holds across composition:
\[
\SigmaType(\muRecon_1 \compose \muRecon_2) = \SigmaType(\muRecon_2)[\SigmaType(\muRecon_1)/\text{input}]
\]

\subsection*{Reconciler \(\muRecon\)}

The reconciler maps filesystem observations to RDF project models:

\[
\muRecon_{\text{scan}}: \Oobs_{\text{fs}} \to \Aout_{\text{model}}
\]

Decomposed into stages:
\begin{enumerate}
\item \texttt{scanFileSystemToStore}: Filesystem $\to$ RDF triples
\item \texttt{detectStackFromFs}: Stack signatures $\to$ Technology ontology
\item \texttt{classifyFiles}: File paths $\to$ Role annotations
\item \texttt{buildProjectModelFromFs}: RDF $\to$ Structured project representation
\end{enumerate}

MAPEK autonomic loop reconciler:
\[
\muRecon_{\text{MAPEK}}: (\text{Monitor}, \text{Analyze}, \text{Plan}, \text{Execute}, \text{Knowledge}) \to \text{Actions}
\]

Implemented via \texttt{runMapekIteration} with knowledge hook integration:
\begin{itemize}
\item Monitor: Collect drift, hotspots, gaps, complexity
\item Analyze: Score violations, rank issues
\item Plan: Generate hook policies, refactoring suggestions
\item Execute: Apply materialization plans
\item Knowledge: Update baseline, record receipts
\end{itemize}

\subsection*{Composition \(\PiMerge / \oplusMerge\)}

Sequential pipeline composition:
\[
\PiMerge(\text{scan}, \text{detect}, \text{classify}, \text{model}) = \text{initialize}
\]

Parallel innovation aggregation:
\[
\Aout_{\text{total}} = \bigoplus_{i \in \text{Innovations}} \muRecon_i(\Oobs_{\text{fs}})
\]

Innovations run concurrently via \texttt{runInnovationsParallel}:
\begin{itemize}
\item Hotspot analysis
\item Gap finding
\item Type auditing
\item Contract validation
\item Complexity analysis
\item Dependency graph analysis
\item Documentation drift checking
\end{itemize}

Results merged via \texttt{aggregateInnovationFindings} with deduplication.

\subsection*{Guard \(\GuardH\) and Invariant \(\InvQ\)}

Guards prevent invalid operations:
\begin{itemize}
\item \(\GuardH_{\text{path}}\): File path must exist before scan
\item \(\GuardH_{\text{baseline}}\): Baseline must exist before drift computation
\item \(\GuardH_{\text{schema}}\): Domain model must validate before materialization
\end{itemize}

Invariants preserved:
\begin{itemize}
\item \(\InvQ_{\text{consistency}}\): RDF model remains consistent with filesystem
\item \(\InvQ_{\text{idempotent}}\): Re-scanning produces equivalent graph
\item \(\InvQ_{\text{snapshot}}\): Snapshot deterministic for given filesystem state
\end{itemize}

Drift detection verifies:
\[
\InvQ_{\text{drift}}(\text{baseline}, \text{current}) \iff \text{computeDrift}(\text{baseline}, \text{current}) = \emptyset
\]

\subsection*{Provenance and Receipts}

MAPEK loop execution produces receipts:
\begin{lstlisting}[language=JavaScript]
{
  timestamp: "2025-12-27T00:00:00.000Z",
  iteration: 42,
  findings: {
    hotspots: [...],
    gaps: [...],
    violations: [...]
  },
  actions: [...],
  hash: "sha256:..."
}
\end{lstlisting}

Provenance chain:
\[
\ProvHash(\text{snapshot}_t) = h(\ProvHash(\text{snapshot}_{t-1}), \Delta_t)
\]

Snapshot serialization enables deterministic reproduction:
\texttt{serializeSnapshot} $\to$ JSON with canonical field ordering.

\subsection*{Minimal Example}

\begin{lstlisting}[language=JavaScript]
import {
  scanFileSystemToStore,
  buildProjectModelFromFs,
  runMapekIteration
} from '@unrdf/project-engine';

// Scan filesystem to RDF
const store = await scanFileSystemToStore('./my-project');

// Build structured model
const model = await buildProjectModelFromFs('./my-project');

// Run autonomic analysis
const result = await runMapekIteration({
  rootPath: './my-project',
  baseline: model,
  enableAllInnovations: true
});

console.log(result.findings.hotspots);
\end{lstlisting}

\subsection*{Open Questions}

\begin{enumerate}
\item Can MAPEK loop termination be guaranteed for cyclic refactoring suggestions?
\item How to handle non-deterministic file modification times in snapshot hashing?
\item What is optimal granularity for hotspot threshold tuning?
\item Can drift detection scale to monorepos with 1000+ packages?
\end{enumerate}

% ============================================================================
% Package 36: unrdf-rdf-graphql
% ============================================================================

\label{pkg:unrdf-rdf-graphql}
\section{\pkg{@unrdf/rdf-graphql} --- Type-Safe GraphQL Interface}

\begin{pkgmeta}
Path & \texttt{packages/rdf-graphql} \\
Kind & js \\
Entrypoints & 4 files \\
Dependencies & 4 \\
Blurb & Type-safe GraphQL interface for RDF knowledge graphs with automatic schema generation \\
\end{pkgmeta}

\subsection*{Observable \(\Oobs\) and Artifact \(\Aout\)}

Observables are RDF ontologies and instance data:

\[
\Oobs_{\text{rdf}} = \{ \text{classes}, \text{properties}, \text{instances}, \text{constraints} \}
\]

Artifacts include GraphQL schemas and query results:

\[
\Aout_{\text{graphql}} = \{ \text{Schema}, \text{Resolvers}, \text{QueryResults} \}
\]

The adapter observes:
\begin{itemize}
\item RDFS/OWL class definitions via SPARQL introspection
\item Property domains and ranges
\item Instance data in triple store
\item GraphQL queries from clients
\end{itemize}

Produces:
\begin{itemize}
\item GraphQL schema definition (SDL)
\item Resolver functions bound to SPARQL queries
\item Query execution results with type safety
\item Cache statistics and performance metrics
\end{itemize}

\subsection*{Type Signature \(\SigmaType\)}

Configuration validated via Zod:

\begin{lstlisting}[language=JavaScript]
const AdapterConfigSchema = z.object({
  namespaces: z.record(z.string()).optional(),
  excludeClasses: z.array(z.string()).optional(),
  includeInferred: z.boolean().optional(),
  enableCache: z.boolean().optional(),
  typeMapping: z.record(z.string()).optional(),
});
\end{lstlisting}

Type mapping from RDF to GraphQL:
\[
\SigmaType_{\text{RDF}}(\text{rdfs:Class}) \mapsto \SigmaType_{\text{GraphQL}}(\text{ObjectType})
\]

Mapping rules:
\begin{itemize}
\item \texttt{owl:Class} $\to$ GraphQL Object Type
\item \texttt{owl:DatatypeProperty} $\to$ GraphQL Field
\item \texttt{owl:ObjectProperty} $\to$ GraphQL Relation
\item \texttt{xsd:string} $\to$ GraphQL String
\item \texttt{xsd:integer} $\to$ GraphQL Int
\end{itemize}

\subsection*{Reconciler \(\muRecon\)}

The reconciler generates GraphQL schemas from RDF ontologies:

\[
\muRecon_{\text{schema}}: \Oobs_{\text{ontology}} \to \Aout_{\text{GraphQLSchema}}
\]

Pipeline stages:
\begin{enumerate}
\item \texttt{RDFSchemaGenerator.loadOntology}: Parse RDF into internal model
\item \texttt{generateSchema}: Traverse ontology, emit GraphQL types
\item \texttt{RDFResolverFactory.createResolvers}: Bind resolvers to SPARQL
\item \texttt{executeQuery}: Route GraphQL queries to SPARQL backend
\end{enumerate}

Query reconciler:
\[
\muRecon_{\text{query}}: \text{GraphQL Query} \to \text{SPARQL Query} \to \text{Results}
\]

\texttt{SPARQLQueryBuilder} translates GraphQL AST to SPARQL:
\begin{itemize}
\item Field selections $\to$ SELECT variables
\item Nested queries $\to$ OPTIONAL patterns
\item Filters $\to$ FILTER expressions
\item Aliases $\to$ AS bindings
\end{itemize}

\subsection*{Composition \(\PiMerge / \oplusMerge\)}

Adapter lifecycle composition:
\[
\PiMerge(\text{load}, \text{generate}, \text{execute}) = \text{Adapter}
\]

Method chaining:
\begin{lstlisting}[language=JavaScript]
await adapter
  .loadOntology(ontologyRDF)
  .generateSchema()
  .executeQuery(query);
\end{lstlisting}

Multiple ontologies merged via commutative union:
\[
\Aout_{\text{schema}} = \bigoplus_{i} \text{loadOntology}(\text{onto}_i)
\]

Schema merging preserves type safety when no conflicts exist.

\subsection*{Guard \(\GuardH\) and Invariant \(\InvQ\)}

Guards enforce preconditions:
\begin{itemize}
\item \(\GuardH_{\text{schema}}\): Schema must be generated before query execution
\item \(\GuardH_{\text{ontology}}\): Ontology must be loaded before schema generation
\item \(\GuardH_{\text{validation}}\): Config must validate against schema
\end{itemize}

Invariants:
\begin{itemize}
\item \(\InvQ_{\text{type}}\): Generated schema respects RDF type constraints
\item \(\InvQ_{\text{cache}}\): Cache returns equivalent results to fresh query
\item \(\InvQ_{\text{introspection}}\): Introspection matches loaded ontology
\end{itemize}

Cache consistency:
\[
\InvQ_{\text{cache}}(\text{query}, t) \iff \text{cached}(\text{query}) = \text{fresh}(\text{query})
\]

\subsection*{Provenance and Receipts}

Query execution traces:
\begin{itemize}
\item GraphQL query input
\item Generated SPARQL query
\item SPARQL execution time
\item Result set size
\item Cache hit/miss status
\end{itemize}

Statistics API provides observability:
\begin{lstlisting}[language=JavaScript]
const stats = adapter.getCacheStats();
// { enabled: true, size: 42 }

const info = await adapter.getStatistics();
// { tripleCount: 1000, classCount: 50, instanceCount: 200 }
\end{lstlisting}

Provenance via introspection:
\texttt{introspectClasses()} returns complete ontology snapshot.

\subsection*{Minimal Example}

\begin{lstlisting}[language=JavaScript]
import { createAdapter } from '@unrdf/rdf-graphql';

const adapter = createAdapter({
  namespaces: { ex: 'http://example.org/' },
  enableCache: true
});

// Load ontology
await adapter.loadOntology(`
  @prefix ex: <http://example.org/> .
  @prefix rdfs: <http://www.w3.org/2000/01/rdf-schema#> .

  ex:Person a rdfs:Class .
  ex:name a rdf:Property ; rdfs:domain ex:Person .
`);

// Generate schema
adapter.generateSchema();

// Query via GraphQL
const result = await adapter.executeQuery(`
  query {
    Person {
      name
    }
  }
`);
\end{lstlisting}

\subsection*{Open Questions}

\begin{enumerate}
\item Can nested GraphQL queries map to efficient SPARQL property paths?
\item How to handle GraphQL mutations with transactional RDF updates?
\item What is cache invalidation strategy for dynamic ontologies?
\item Can schema generation handle OWL reasoning and inference?
\end{enumerate}

% ============================================================================
% Package 37: unrdf-semantic-search
% ============================================================================

\label{pkg:unrdf-semantic-search}
\section{\pkg{@unrdf/semantic-search} --- AI-Powered Vector Search}

\begin{pkgmeta}
Path & \texttt{packages/semantic-search} \\
Kind & js \\
Entrypoints & 4 files \\
Dependencies & 5 \\
Blurb & AI-powered search over RDF knowledge graphs using vector embeddings \\
\end{pkgmeta}

\subsection*{Observable \(\Oobs\) and Artifact \(\Aout\)}

Observables are RDF entities and natural language queries:

\[
\Oobs_{\text{search}} = \{ \text{entities}, \text{predicates}, \text{literals}, \text{queries}_{\text{NL}} \}
\]

Artifacts include vector embeddings and ranked results:

\[
\Aout_{\text{search}} = \{ \text{embeddings}, \text{index}, \text{results}, \text{scores} \}
\]

The engine observes:
\begin{itemize}
\item RDF entity descriptions (labels, comments, related entities)
\item User natural language queries
\item Vector similarity computations
\item Knowledge graph topology
\end{itemize}

Produces:
\begin{itemize}
\item Entity embeddings via transformer models
\item Vector index for approximate nearest neighbor search
\item Ranked search results with similarity scores
\item Entity recommendations based on graph structure
\end{itemize}

\subsection*{Type Signature \(\SigmaType\)}

Component signatures:

\[
\SigmaType_{\text{embedder}}: \text{RDFEntity} \to \mathbb{R}^{384}
\]

Using \texttt{@xenova/transformers} with MiniLM model:
\begin{lstlisting}[language=JavaScript]
// RDFEmbedder generates 384-dim vectors
class RDFEmbedder {
  async embed(entity) {
    // entity: { uri, label, comment, properties }
    const text = this.serializeEntity(entity);
    const embedding = await this.model.encode(text);
    return embedding; // Float32Array[384]
  }
}
\end{lstlisting}

Search engine signature:
\[
\SigmaType_{\text{search}}: \text{Query}_{\text{NL}} \to [\text{Entity} \times \mathbb{R}]^*
\]

Returns ranked list of (entity, score) pairs.

\subsection*{Reconciler \(\muRecon\)}

The reconciler maps natural language to RDF entities:

\[
\muRecon_{\text{search}}: \Oobs_{\text{query}} \to \Aout_{\text{results}}
\]

Search pipeline:
\begin{enumerate}
\item \texttt{RDFEmbedder.embed(query)}: Query $\to$ embedding vector
\item \texttt{VectorIndex.search(vector, k)}: Find k-nearest neighbors
\item \texttt{SemanticQueryEngine.rerank(results)}: Apply SPARQL filtering
\item \texttt{KnowledgeRecommender.expand(entities)}: Find related entities
\end{enumerate}

Hybrid search reconciler combines vector and graph:
\[
\muRecon_{\text{hybrid}} = \lambda q. \alpha \cdot \text{vector}(q) + (1-\alpha) \cdot \text{sparql}(q)
\]

Where $\alpha \in [0,1]$ balances similarity vs. structured query.

\subsection*{Composition \(\PiMerge / \oplusMerge\)}

Multi-stage search pipeline:
\[
\PiMerge(\text{embed}, \text{search}, \text{rerank}, \text{expand}) = \text{QueryEngine}
\]

Index construction merges entity embeddings:
\[
\text{Index} = \bigoplus_{e \in \text{Entities}} \text{embed}(e)
\]

Using Vectra library for vector storage with commutative insertion.

\subsection*{Guard \(\GuardH\) and Invariant \(\InvQ\)}

Guards:
\begin{itemize}
\item \(\GuardH_{\text{model}}\): Transformer model must load before embedding
\item \(\GuardH_{\text{index}}\): Index must be built before search
\item \(\GuardH_{\text{dim}}\): All vectors must have dimension 384
\end{itemize}

Invariants:
\begin{itemize}
\item \(\InvQ_{\text{symmetric}}\): Distance metric is symmetric
\item \(\InvQ_{\text{topk}}\): Search returns exactly k results (if available)
\item \(\InvQ_{\text{score}}\): Scores monotonically decrease in result list
\end{itemize}

Cosine similarity invariant:
\[
\InvQ_{\text{cosine}}(u, v) = \frac{u \cdot v}{\|u\| \|v\|} \in [-1, 1]
\]

\subsection*{Provenance and Receipts}

Search execution trace:
\begin{itemize}
\item Input query text
\item Generated embedding vector (first 10 dimensions logged)
\item Vector search time (ms)
\item SPARQL reranking time (ms)
\item Result URIs and scores
\end{itemize}

Receipts enable reproducibility:
\begin{lstlisting}[language=JavaScript]
{
  query: "artificial intelligence papers",
  embedding: [0.123, -0.456, ...],
  results: [
    { uri: "ex:Paper1", score: 0.92 },
    { uri: "ex:Paper2", score: 0.87 }
  ],
  timing: { vector: 12, sparql: 8 }
}
\end{lstlisting}

\subsection*{Minimal Example}

\begin{lstlisting}[language=JavaScript]
import {
  RDFEmbedder,
  SemanticQueryEngine
} from '@unrdf/semantic-search';

const embedder = new RDFEmbedder();
await embedder.initialize();

const engine = new SemanticQueryEngine({
  store: myStore,
  embedder: embedder
});

// Index entities
await engine.indexEntities();

// Search
const results = await engine.search(
  "machine learning research",
  { topK: 10 }
);

console.log(results.map(r => r.uri));
\end{lstlisting}

\subsection*{Open Questions}

\begin{enumerate}
\item Can incremental index updates maintain search quality?
\item How to balance vector similarity vs. SPARQL precision?
\item What is optimal entity serialization for embedding?
\item Can multi-modal embeddings (text + graph structure) improve results?
\end{enumerate}

% ============================================================================
% Package 38: unrdf-serverless
% ============================================================================

\label{pkg:unrdf-serverless}
\section{\pkg{@unrdf/serverless} --- AWS Deployment Toolkit}

\begin{pkgmeta}
Path & \texttt{packages/serverless} \\
Kind & js \\
Entrypoints & 5 files \\
Dependencies & 10 \\
Blurb & One-click AWS deployment for RDF applications \\
\end{pkgmeta}

\subsection*{Observable \(\Oobs\) and Artifact \(\Aout\)}

Observables are application code and infrastructure requirements:

\[
\Oobs_{\text{deploy}} = \{ \text{handlers}, \text{config}, \text{dependencies}, \text{env} \}
\]

Artifacts include AWS CloudFormation stacks and deployed resources:

\[
\Aout_{\text{infra}} = \{ \text{Lambdas}, \text{API Gateway}, \text{DynamoDB}, \text{CDN} \}
\]

The toolkit observes:
\begin{itemize}
\item Lambda handler code (MJS files)
\item API endpoint configurations
\item CDK stack definitions
\item Environment variables and secrets
\end{itemize}

Produces:
\begin{itemize}
\item Bundled Lambda functions (esbuild output)
\item CloudFormation templates
\item Deployed API Gateway endpoints
\item DynamoDB tables for RDF storage
\end{itemize}

\subsection*{Type Signature \(\SigmaType\)}

Stack configuration type:
\begin{lstlisting}[language=JavaScript]
const StackConfigSchema = z.object({
  stackName: z.string(),
  environment: z.enum(['dev', 'staging', 'prod']),
  region: z.enum(SUPPORTED_REGIONS),
  memorySizeMb: z.number().min(128).max(10240),
  timeoutSeconds: z.number().min(1).max(900),
  enableCdn: z.boolean().default(false)
});
\end{lstlisting}

API Gateway endpoint signature:
\[
\SigmaType_{\text{endpoint}}: (\text{Path}, \text{Method}, \text{Handler}) \to \text{Route}
\]

DynamoDB adapter type:
\[
\SigmaType_{\text{storage}}: \text{RDF Quad} \to \text{DynamoDB Item}
\]

\subsection*{Reconciler \(\muRecon\)}

The reconciler transforms application code to deployed infrastructure:

\[
\muRecon_{\text{deploy}}: \Oobs_{\text{app}} \to \Aout_{\text{AWS}}
\]

Deployment pipeline:
\begin{enumerate}
\item \texttt{LambdaBundler.bundle()}: MJS $\to$ Optimized bundle
\item \texttt{UNRDFStack.addLambda()}: Bundle $\to$ CDK construct
\item \texttt{ApiGatewayConfig.addEndpoint()}: Handler $\to$ API route
\item \texttt{cdk deploy}: CDK $\to$ CloudFormation $\to$ AWS resources
\end{enumerate}

Request reconciler (runtime):
\[
\muRecon_{\text{request}}: \text{HTTP Request} \to \text{Lambda} \to \text{HTTP Response}
\]

Via API Gateway integration with validation middleware.

\subsection*{Composition \(\PiMerge / \oplusMerge\)}

CDK stack composition:
\[
\PiMerge(\text{Lambdas}, \text{API}, \text{Storage}, \text{CDN}) = \text{UNRDFStack}
\]

Sequential deployment stages:
\begin{lstlisting}[language=JavaScript]
const stack = createUNRDFStack(app, 'MyApp', config);
stack
  .addLambda('queryHandler', './dist/query.js')
  .addLambda('updateHandler', './dist/update.js')
  .addApiGateway(apiConfig)
  .addDynamoTable('triples');
\end{lstlisting}

Multi-region deployment via parallel stacks:
\[
\Aout_{\text{global}} = \bigoplus_{r \in \text{Regions}} \text{deploy}(r)
\]

\subsection*{Guard \(\GuardH\) and Invariant \(\InvQ\)}

Guards enforce AWS limits:
\begin{itemize}
\item \(\GuardH_{\text{memory}}\): Lambda memory 128MB--10GB
\item \(\GuardH_{\text{timeout}}\): Lambda timeout 1--900 seconds
\item \(\GuardH_{\text{bundle}}\): Bundle size $<$ 250MB uncompressed
\item \(\GuardH_{\text{region}}\): Region in \texttt{SUPPORTED\_REGIONS}
\end{itemize}

Invariants:
\begin{itemize}
\item \(\InvQ_{\text{idempotent}}\): Re-deploying same code produces equivalent stack
\item \(\InvQ_{\text{storage}}\): DynamoDB round-trip preserves RDF quad structure
\item \(\InvQ_{\text{api}}\): All endpoints return valid HTTP status codes
\end{itemize}

Request validation invariant:
\[
\InvQ_{\text{validate}}(\text{req}) \iff \text{validateApiRequest}(\text{req}) \neq \text{error}
\]

\subsection*{Provenance and Receipts}

Deployment receipt:
\begin{lstlisting}[language=JavaScript]
{
  stackName: "UnrdfProd",
  region: "us-east-1",
  deploymentTime: "2025-12-27T12:00:00Z",
  resources: {
    lambdas: ["queryHandler-AbC123"],
    apis: ["https://api.example.com/v1"],
    tables: ["triples-prod"]
  },
  bundleHashes: {
    "queryHandler": "sha256:abc...",
    "updateHandler": "sha256:def..."
  }
}
\end{lstlisting}

CloudFormation stack provides infrastructure provenance via AWS APIs.

\subsection*{Minimal Example}

\begin{lstlisting}[language=JavaScript]
import {
  createUNRDFStack,
  LambdaBundler
} from '@unrdf/serverless';
import { App } from 'aws-cdk-lib';

const app = new App();

// Bundle functions
const bundler = new LambdaBundler({
  entryPoint: './src/handler.mjs',
  outDir: './dist'
});
await bundler.bundle();

// Create stack
const stack = createUNRDFStack(app, 'MyRDFApp', {
  environment: 'prod',
  memorySizeMb: 2048,
  enableCdn: true
});

// Deploy: cdk deploy
\end{lstlisting}

\subsection*{Open Questions}

\begin{enumerate}
\item Can Lambda cold starts be mitigated for large RDF graphs?
\item How to handle SPARQL query optimization in DynamoDB backend?
\item What is cost-optimal sharding strategy for triple storage?
\item Can CDK construct library support multi-cloud (Azure, GCP)?
\end{enumerate}

% ============================================================================
% Package 39: unrdf-streaming
% ============================================================================

\label{pkg:unrdf-streaming}
\section{\pkg{@unrdf/streaming} --- Real-Time Change Feeds}

\begin{pkgmeta}
Path & \texttt{packages/streaming} \\
Kind & js \\
Entrypoints & 2 files \\
Dependencies & 10 \\
Blurb & Change Feeds and Real-time Synchronization \\
\end{pkgmeta}

\subsection*{Observable \(\Oobs\) and Artifact \(\Aout\)}

Observables are RDF store mutations and subscriptions:

\[
\Oobs_{\text{stream}} = \{ \Delta_{\text{add}}, \Delta_{\text{remove}}, \text{subscriptions} \}
\]

Artifacts include change events and synchronized replicas:

\[
\Aout_{\text{stream}} = \{ \text{events}, \text{checksums}, \text{replicas} \}
\]

The streaming engine observes:
\begin{itemize}
\item Quad additions and removals in store
\item Client subscription patterns (SPARQL filters)
\item WebSocket connections for real-time delivery
\item Synchronization protocol messages
\end{itemize}

Produces:
\begin{itemize}
\item Change feed event streams
\item Subscription notifications matching patterns
\item Sync protocol messages with checksums
\item Conflict-free replica convergence
\end{itemize}

\subsection*{Type Signature \(\SigmaType\)}

Change event type:
\begin{lstlisting}[language=JavaScript]
const ChangeEventSchema = z.object({
  type: z.enum(['add', 'remove']),
  quad: QuadSchema,
  timestamp: z.number(),
  source: z.string()
});
\end{lstlisting}

Subscription signature:
\[
\SigmaType_{\text{sub}}: \text{Pattern} \to \text{Stream}[\text{ChangeEvent}]
\]

Sync message type:
\begin{lstlisting}[language=JavaScript]
const SyncMessageSchema = z.object({
  sequence: z.number(),
  changes: z.array(ChangeEventSchema),
  checksum: z.string(),
  acknowledged: z.boolean()
});
\end{lstlisting}

\subsection*{Reconciler \(\muRecon\)}

The reconciler transforms store mutations into stream events:

\[
\muRecon_{\text{feed}}: \Delta_{\text{store}} \to \text{Stream}[\text{ChangeEvent}]
\]

Change feed pipeline:
\begin{enumerate}
\item Store hook intercepts quad operations
\item \texttt{createChangeFeed} emits events
\item \texttt{SubscriptionManager} filters by pattern
\item \texttt{StreamProcessor} delivers to clients
\end{enumerate}

Sync reconciler merges remote changes:
\[
\muRecon_{\text{sync}}: \text{SyncMessage} \to \store_{\text{local}}
\]

Conflict resolution via causal ordering:
\begin{itemize}
\item Sequence numbers establish total order
\item Checksums detect divergence
\item \texttt{mergeSyncMessages} applies commutative merge
\end{itemize}

\subsection*{Composition \(\PiMerge / \oplusMerge\)}

Feed subscription composition:
\[
\PiMerge(\text{feed}, \text{filter}, \text{transform}, \text{deliver}) = \text{Subscription}
\]

Multiple subscriptions merged:
\[
\text{Stream}_{\text{combined}} = \bigoplus_{i} \text{subscribe}(\text{pattern}_i)
\]

WebSocket multiplexing delivers merged stream over single connection.

\subsection*{Guard \(\GuardH\) and Invariant \(\InvQ\)}

Guards:
\begin{itemize}
\item \(\GuardH_{\text{connected}}\): Client must connect before subscription
\item \(\GuardH_{\text{pattern}}\): Subscription pattern must be valid SPARQL
\item \(\GuardH_{\text{sequence}}\): Sync messages must have monotonic sequence
\end{itemize}

Invariants:
\begin{itemize}
\item \(\InvQ_{\text{delivery}}\): All matching events delivered exactly once
\item \(\InvQ_{\text{order}}\): Events delivered in causal order
\item \(\InvQ_{\text{checksum}}\): Checksums verify message integrity
\item \(\InvQ_{\text{convergence}}\): Replicas converge given all sync messages
\end{itemize}

Checksum invariant:
\[
\InvQ_{\text{checksum}}(\text{msg}) \iff \text{calculateChecksum}(\text{msg}) = \text{msg.checksum}
\]

Convergence theorem:
\[
\InvQ_{\text{conv}}(\store_1, \store_2) \iff \text{applyAll}(\text{sync}) \Rightarrow \store_1 \equiv \store_2
\]

\subsection*{Provenance and Receipts}

Event stream receipt:
\begin{lstlisting}[language=JavaScript]
{
  subscriptionId: "sub-123",
  pattern: "?s rdf:type ex:Event",
  eventCount: 42,
  firstEvent: "2025-12-27T12:00:00.000Z",
  lastEvent: "2025-12-27T12:05:00.000Z",
  deliveryLatency: { p50: 12, p99: 45 }
}
\end{lstlisting}

Sync protocol audit trail:
\begin{itemize}
\item Sequence numbers enable gap detection
\item Checksums provide tamper evidence
\item Acknowledgments confirm delivery
\end{itemize}

\subsection*{Minimal Example}

\begin{lstlisting}[language=JavaScript]
import {
  createChangeFeed,
  createSubscriptionManager
} from '@unrdf/streaming';

const feed = createChangeFeed(store);
const subManager = createSubscriptionManager(feed);

// Subscribe to pattern
const sub = subManager.subscribe({
  pattern: '?s rdf:type ex:Person',
  callback: (event) => {
    console.log('Change:', event.type, event.quad);
  }
});

// Add quad triggers notification
store.add(quad);
\end{lstlisting}

\subsection*{Open Questions}

\begin{enumerate}
\item Can causal broadcast scale to 1000+ concurrent subscribers?
\item How to handle late joiners with complete history replay?
\item What is optimal batch size for sync message aggregation?
\item Can conflict-free replicated data types (CRDTs) improve convergence?
\end{enumerate}

% ============================================================================
% Package 40: unrdf-test-utils
% ============================================================================

\label{pkg:unrdf-test-utils}
\section{\pkg{@unrdf/test-utils} --- Testing Utilities}

\begin{pkgmeta}
Path & \texttt{packages/test-utils} \\
Kind & js \\
Entrypoints & 1 file \\
Dependencies & 4 \\
Blurb & Testing utilities for UNRDF development \\
\end{pkgmeta}

\subsection*{Observable \(\Oobs\) and Artifact \(\Aout\)}

Observables are test execution traces and assertions:

\[
\Oobs_{\text{test}} = \{ \text{scenarios}, \text{steps}, \text{contexts}, \text{assertions} \}
\]

Artifacts include test results and execution reports:

\[
\Aout_{\text{test}} = \{ \text{results}, \text{failures}, \text{durations}, \text{coverage} \}
\]

The framework observes:
\begin{itemize}
\item Test scenario definitions (DSL)
\item Step execution outcomes
\item Assertion evaluations
\item Store state mutations
\end{itemize}

Produces:
\begin{itemize}
\item Test scenario execution results
\item Fluent assertion chains
\item Test context builders
\item Mock receipt and hook fixtures
\end{itemize}

\subsection*{Type Signature \(\SigmaType\)}

Test scenario schema:
\begin{lstlisting}[language=JavaScript]
const TestScenarioSchema = z.object({
  name: z.string().min(1),
  description: z.string().optional(),
  setup: z.function().optional().nullable(),
  teardown: z.function().optional().nullable(),
  steps: z.array(z.object({
    name: z.string().min(1),
    action: z.function(),
    assertions: z.array(z.function()).optional()
  })).min(1)
});
\end{lstlisting}

Assertion signature:
\[
\SigmaType_{\text{assert}}: (\text{Context}, \text{Result}) \to \text{Boolean} \cup \text{Error}
\]

Test context type:
\[
\SigmaType_{\text{ctx}} = \{ \store, \text{manager}, \text{metadata} \}
\]

\subsection*{Reconciler \(\muRecon\)}

The reconciler executes test scenarios and produces results:

\[
\muRecon_{\text{test}}: \Oobs_{\text{scenario}} \to \Aout_{\text{result}}
\]

Execution pipeline:
\begin{enumerate}
\item \texttt{TestScenario.setupScenario()}: Initialize context
\item \texttt{TestScenario.step()}: Execute actions sequentially
\item \texttt{FluentAssertions.*}: Validate results
\item \texttt{TestScenario.teardownScenario()}: Cleanup
\end{enumerate}

Fluent assertion reconciler:
\[
\muRecon_{\text{assert}} = \lambda (\text{ctx}, \text{res}). \text{chain}(\text{assertions})
\]

Supports method chaining:
\begin{lstlisting}[language=JavaScript]
expect(ctx, result)
  .toBeCommitted()
  .toHaveProperty('receipt')
  .toCompleteWithin(100);
\end{lstlisting}

\subsection*{Composition \(\PiMerge / \oplusMerge\)}

Test scenario composition:
\[
\PiMerge(\text{setup}, \text{steps}, \text{assertions}, \text{teardown}) = \text{Scenario}
\]

Step-wise sequential execution:
\begin{lstlisting}[language=JavaScript]
scenario("Knowledge Hook Test")
  .setupScenario(async () => ({ store, manager }))
  .step("Add quad", async (ctx) => { ... })
  .step("Execute hook", async (ctx) => { ... })
  .teardownScenario(async (ctx) => { ... });
\end{lstlisting}

Multiple assertions merged:
\[
\Aout_{\text{valid}} = \bigwedge_{i} \text{assertion}_i(\text{ctx}, \text{res})
\]

All assertions must pass for step success.

\subsection*{Guard \(\GuardH\) and Invariant \(\InvQ\)}

Guards:
\begin{itemize}
\item \(\GuardH_{\text{steps}}\): Scenario must have at least one step
\item \(\GuardH_{\text{context}}\): Context must be initialized before steps
\item \(\GuardH_{\text{schema}}\): Scenario must validate against schema
\end{itemize}

Invariants:
\begin{itemize}
\item \(\InvQ_{\text{isolation}}\): Tests do not share state between scenarios
\item \(\InvQ_{\text{teardown}}\): Teardown always executes (even on failure)
\item \(\InvQ_{\text{deterministic}}\): Re-running scenario produces same result
\end{itemize}

Isolation invariant:
\[
\InvQ_{\text{iso}}(\text{test}_1, \text{test}_2) \iff \text{ctx}_1 \cap \text{ctx}_2 = \emptyset
\]

\subsection*{Provenance and Receipts}

Test execution receipt:
\begin{lstlisting}[language=JavaScript]
{
  name: "Hook Execution Test",
  success: true,
  steps: [
    { name: "Add quad", success: true, duration: 5 },
    { name: "Execute hook", success: true, duration: 12 }
  ],
  errors: [],
  duration: 23
}
\end{lstlisting}

Provides complete audit trail for test failures.

\subsection*{Minimal Example}

\begin{lstlisting}[language=JavaScript]
import { scenario, expect, TestHelpers } from '@unrdf/test-utils';

const test = scenario("Add quad and verify", "Test store operations")
  .setupScenario(async () => createDefaultTestContext())
  .step("Add quad", async (ctx) => {
    const quad = TestHelpers.createQuad(
      'ex:s', 'ex:p', 'ex:o'
    );
    ctx.store.add(quad);
    return { quad };
  }, [
    (ctx, result) => expect(ctx, result)
      .toHaveProperty('quad')
      .toContainQuads([result.quad])
  ])
  .teardownScenario(async (ctx) => {
    ctx.store.clear();
  });

const result = await test.execute();
console.log(result.success);
\end{lstlisting}

\subsection*{Open Questions}

\begin{enumerate}
\item Can test scenarios compose to form integration test suites?
\item How to handle non-deterministic timing in OTEL span assertions?
\item What is optimal granularity for test context fixtures?
\item Can property-based testing integrate with scenario DSL?
\end{enumerate}

% ============================================================================
% Package 41: unrdf-validation
% ============================================================================

\label{pkg:unrdf-validation}
\section{\pkg{@unrdf/validation} --- OTEL Validation Framework}

\begin{pkgmeta}
Path & \texttt{packages/validation} \\
Kind & js \\
Entrypoints & 1 file \\
Dependencies & 1 \\
Blurb & OTEL validation framework for UNRDF development \\
\end{pkgmeta}

\subsection*{Observable \(\Oobs\) and Artifact \(\Aout\)}

Observables are OpenTelemetry spans and traces:

\[
\Oobs_{\text{otel}} = \{ \text{spans}, \text{attributes}, \text{events}, \text{traces} \}
\]

Artifacts include validation scores and failure reports:

\[
\Aout_{\text{valid}} = \{ \text{scores}, \text{violations}, \text{metrics}, \text{receipts} \}
\]

The validator observes:
\begin{itemize}
\item OTEL span creation and completion
\item Span attributes and context
\item Trace hierarchies and timing
\item Custom events and annotations
\end{itemize}

Produces:
\begin{itemize}
\item Validation scores (0--100)
\item Violation reports with evidence
\item Performance metrics from spans
\item Receipts linking code to traces
\end{itemize}

\subsection*{Type Signature \(\SigmaType\)}

Validation configuration:
\begin{lstlisting}[language=JavaScript]
const ValidationConfigSchema = z.object({
  minScore: z.number().min(0).max(100).default(80),
  requiredSpans: z.array(z.string()).optional(),
  maxDuration: z.number().optional(),
  attributeValidators: z.record(z.function()).optional()
});
\end{lstlisting}

Validator signature:
\[
\SigmaType_{\text{validator}}: \text{Trace} \to [\text{Score}, \text{Violations}]
\]

Helper signature:
\[
\SigmaType_{\text{helper}}: \text{Span} \to \text{Metric}
\]

\subsection*{Reconciler \(\muRecon\)}

The reconciler validates execution traces against requirements:

\[
\muRecon_{\text{validate}}: \Oobs_{\text{trace}} \to \Aout_{\text{score}}
\]

Validation pipeline:
\begin{enumerate}
\item \texttt{OTELValidator.collectSpans()}: Gather all spans
\item \texttt{ValidationHelpers.extractMetrics()}: Compute span metrics
\item \texttt{ValidationRunner.scoreTrace()}: Apply validation rules
\item \texttt{ValidationRunner.reportViolations()}: Generate report
\end{enumerate}

Score reconciler:
\[
\muRecon_{\text{score}} = \lambda \text{trace}. \frac{\sum_{r \in \text{Rules}} w_r \cdot \text{check}(r, \text{trace})}{\sum_{r} w_r}
\]

Where $w_r$ are rule weights, normalized to [0,100].

\subsection*{Composition \(\PiMerge / \oplusMerge\)}

Multi-rule validation composition:
\[
\PiMerge(\text{structure}, \text{timing}, \text{attributes}, \text{events}) = \text{Validator}
\]

Parallel validation across traces:
\[
\Aout_{\text{suite}} = \bigoplus_{t \in \text{Traces}} \text{validate}(t)
\]

Aggregate scores via weighted average.

\subsection*{Guard \(\GuardH\) and Invariant \(\InvQ\)}

Guards:
\begin{itemize}
\item \(\GuardH_{\text{trace}}\): Trace must be complete before validation
\item \(\GuardH_{\text{spans}}\): All required spans must exist
\item \(\GuardH_{\text{timing}}\): Span end time $\geq$ start time
\end{itemize}

Invariants:
\begin{itemize}
\item \(\InvQ_{\text{score}}\): Score in [0, 100]
\item \(\InvQ_{\text{monotonic}}\): Adding spans cannot increase score
\item \(\InvQ_{\text{deterministic}}\): Same trace produces same score
\end{itemize}

Timing invariant:
\[
\InvQ_{\text{timing}}(\text{span}) \iff \text{span.endTime} \geq \text{span.startTime}
\]

Determinism:
\[
\InvQ_{\text{det}}(t) \iff \text{validate}(t)_1 = \text{validate}(t)_2
\]

\subsection*{Provenance and Receipts}

Validation receipt:
\begin{lstlisting}[language=JavaScript]
{
  traceId: "abc123...",
  score: 92,
  violations: [
    {
      rule: "max-duration",
      severity: "warning",
      span: "compute-embedding",
      expected: 100,
      actual: 142
    }
  ],
  metrics: {
    totalSpans: 15,
    avgDuration: 45,
    maxDepth: 4
  },
  timestamp: "2025-12-27T12:00:00.000Z"
}
\end{lstlisting}

Receipt hash enables verification:
\[
\ProvHash(\text{receipt}) = h(\text{traceId}, \text{score}, \text{violations})
\]

\subsection*{Minimal Example}

\begin{lstlisting}[language=JavaScript]
import {
  createOTELValidator,
  createValidationRunner
} from '@unrdf/validation';

const validator = createOTELValidator({
  requiredSpans: ['query', 'parse', 'execute']
});

const runner = createValidationRunner({
  minScore: 80,
  maxDuration: 1000
});

// Run validation
const result = await runner.validate(trace);

console.log(`Score: ${result.score}/100`);
console.log(`Violations: ${result.violations.length}`);
\end{lstlisting}

\subsection*{Open Questions}

\begin{enumerate}
\item Can validation rules be learned from historical trace data?
\item How to handle distributed traces spanning multiple services?
\item What is optimal weight assignment for heterogeneous rule sets?
\item Can validation integrate with CI/CD for automated quality gates?
\end{enumerate}

% ============================================================================
% End of Agent 8 Packages
% ============================================================================
