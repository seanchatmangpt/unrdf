% preamble.tex - KGC Substrate Notation + Metadata Infrastructure
% Knowledge Graph Compiler notation for @unrdf ecosystem thesis

% ============================================================================
% KGC Substrate Notation Macros
% ============================================================================

% Core substrate operators
\newcommand{\Oobs}{\mathcal{O}}          % Observable substrate (data source, event stream)
\newcommand{\Aout}{\mathcal{A}}          % Artifact output (triple store, document, proof)
\newcommand{\muRecon}{\mu}               % Reconciler function (O → A)
\newcommand{\SigmaType}{\Sigma}          % Type/schema signature (Zod, JSON-LD context)
\newcommand{\PiMerge}{\Pi}               % Merge operator (composition, sequential)
\newcommand{\oplusMerge}{\oplus}         % Merge glue operator (commutative fusion)
\newcommand{\tauEpoch}{\tau}             % Epoch / time / causality marker
\newcommand{\GammaGlue}{\Gamma}          % Glue / composition constraint
\newcommand{\GuardH}{H}                  % Guard (impossibility predicate)
\newcommand{\InvQ}{Q}                    % Invariant preservation predicate
\newcommand{\ProvHash}{\mathtt{hash}}    % Provenance hash (content-addressing)

% Package-level identifiers
\newcommand{\pkg}[1]{\texttt{#1}}        % Package name in monospace
\newcommand{\pkgver}[2]{\texttt{#1}@\texttt{#2}}  % Package@version

% ============================================================================
% Metadata Table Environment
% ============================================================================

% Usage:
%   \begin{pkgmeta}
%     \hline
%     Property & Value \\
%     \hline
%     Package & \pkg{@unrdf/core} \\
%     Version & 1.0.0 \\
%     ...
%   \end{pkgmeta}

\newenvironment{pkgmeta}
  {\begin{table}[h]\centering\begin{tabular}{|l|l|}\hline}
  {\hline\end{tabular}\caption{Package Metadata}\end{table}}

% ============================================================================
% RDF-Specific Notation
% ============================================================================

\newcommand{\triple}[3]{\langle #1, #2, #3 \rangle}  % RDF triple <s, p, o>
\newcommand{\quad}[4]{\langle #1, #2, #3, #4 \rangle}  % RDF quad <s, p, o, g>
\newcommand{\store}{\mathcal{S}}                     % Triple store instance
\newcommand{\query}[1]{\mathtt{SPARQL}(#1)}          % SPARQL query
\newcommand{\prefix}[2]{\texttt{#1:}#2}              % Namespace prefix

% ============================================================================
% Compositional Semantics
% ============================================================================

% Function composition: f ∘ g
\newcommand{\compose}{\circ}

% Type judgment: x : T
\newcommand{\oftype}[2]{#1 : #2}

% Entailment: Γ ⊢ Q
\newcommand{\entails}{\vdash}

% Contradiction (impossibility)
\newcommand{\impossible}{\bot}

% Tautology (always holds)
\newcommand{\necessary}{\top}

% ============================================================================
% Receipt-Based Verification
% ============================================================================

\newcommand{\receipt}[1]{\textsc{Receipt}(#1)}       % Receipt reference
\newcommand{\proof}[1]{\textsc{Proof}(#1)}           % Proof reference
\newcommand{\measure}[1]{\textsc{Measure}(#1)}       % Measurement result

% ============================================================================
% Standard Packages
% ============================================================================

\usepackage{amsmath}      % Mathematical typesetting
\usepackage{amssymb}      % Mathematical symbols
\usepackage{amsthm}       % Theorem environments
\usepackage{listings}     % Code listings
\usepackage{xcolor}       % Color support
\usepackage{graphicx}     % Graphics inclusion
\usepackage{booktabs}     % Professional tables
\usepackage{hyperref}     % Hyperlinks (load last)

% ============================================================================
% Theorem Environments
% ============================================================================

\theoremstyle{definition}
\newtheorem{definition}{Definition}[chapter]
\newtheorem{property}{Property}[chapter]
\newtheorem{invariant}{Invariant}[chapter]
\newtheorem{constraint}{Constraint}[chapter]

\theoremstyle{plain}
\newtheorem{theorem}{Theorem}[chapter]
\newtheorem{lemma}[theorem]{Lemma}
\newtheorem{corollary}[theorem]{Corollary}

\theoremstyle{remark}
\newtheorem{remark}{Remark}[chapter]
\newtheorem{example}{Example}[chapter]

% ============================================================================
% Code Listing Configuration
% ============================================================================

\lstdefinestyle{javascript}{
    language=JavaScript,
    basicstyle=\ttfamily\small,
    keywordstyle=\color{blue}\bfseries,
    commentstyle=\color{gray}\itshape,
    stringstyle=\color{red},
    numbers=left,
    numberstyle=\tiny\color{gray},
    stepnumber=1,
    numbersep=5pt,
    backgroundcolor=\color{white},
    showspaces=false,
    showstringspaces=false,
    showtabs=false,
    frame=single,
    tabsize=2,
    captionpos=b,
    breaklines=true,
    breakatwhitespace=false,
    escapeinside={(*@}{@*)},
}

\lstset{style=javascript}

% ============================================================================
% Hyperref Configuration
% ============================================================================

\hypersetup{
    colorlinks=true,
    linkcolor=blue,
    citecolor=blue,
    urlcolor=blue,
    pdftitle={@unrdf Ecosystem: A Compositional Foundation for RDF Processing},
    pdfauthor={Automated PhD-Thesis Documentation System},
    pdfsubject={RDF Processing, Knowledge Graphs, Compositional Systems},
    pdfkeywords={RDF, Oxigraph, Knowledge Graph Compiler, Compositional Semantics}
}

% ============================================================================
% Page Layout
% ============================================================================

\usepackage[margin=1in]{geometry}
\usepackage{setspace}
\onehalfspacing

% ============================================================================
% End of preamble.tex
% ============================================================================
